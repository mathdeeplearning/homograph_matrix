\subsection{反称矩阵}
 	向量$\mathbf{u},\mathbf{v}$的叉乘,
 	\begin{align*}
 		\mathbf{u}\times \mathbf{v} &= 
 		\begin{vmatrix}
 			\mathbf{i}\quad & \mathbf{j}\quad & \mathbf{k}\\
 			u_1\quad& u_2\quad& u_3\\
 			v_1\quad& v_2\quad& v_3
 		\end{vmatrix}\\
 		&= (u_2v_3 -u_3v_2)\mathbf{i} + (u_3v_1 - u_1v_3)\mathbf{j} +(u_1v_2 - u_2v_1)\mathbf{k}\\
 		&=\begin{bmatrix}
 			0 \quad& -u_3\quad & u_2\\
 			u_3\quad & 0\quad& -u_1\\
 			-u_2\quad& u_1\quad& 0
 		\end{bmatrix}
 		\begin{bmatrix}
 			v_1\\
 			v_2\\
 			v_3
 		\end{bmatrix}
 		\\
 		&= [\mathbf{u}]_{\times} \cdot \mathbf{v}
 	\end{align*}

 	$[\mathbf{u}]_{\times}$为由$\mathbf{u}$张成的\textbf{反称矩阵},
 	$$
 		[\mathbf{u}]_{\times} = 
 		\begin{bmatrix}
 			0 \quad& -u_3\quad & u_2\\
 			u_3\quad & 0\quad& -u_1\\
 			-u_2\quad& u_1\quad& 0
 		\end{bmatrix}
 	$$

	存在如下性质,

	\begin{itemize}
		\item 对任意向量$p$,$p^T [\mathbf{u}]_{\times} p = 0$
		\item 矩阵秩为2,特征值为和纯虚数;所以基础矩阵秩为2
		\item 二次幂可展开为,
			$$
				[\mathbf{u}]_{\times}^2 = \mathbf{u}\mathbf{u}^T - \Vert\mathbf{u}\Vert^2 \mathbf{I_3}
			$$
		\item 三次幂等价于自身,
			$$
				[\mathbf{u}]_{\times}^3 = -\Vert\mathbf{u}\Vert^2[\mathbf{u}]_{\times}
			$$
		\item 正交分解,
			$$
				[\mathbf{u}]_{\times} = kUZU^T
			$$

			$U$是单位正交矩阵,$k$是一个尺度因子,$\mathbf{u}$如果是齐次坐标,可以忽略;$Z$为,
			$$
				Z = \begin{bmatrix}
					0 \quad& 1\quad& 0\\
					-1 \quad& 0\quad& 0\\
					0 \quad& 0\quad& 0
				\end{bmatrix}
				\quad
				 W = \begin{bmatrix}
					0 \quad& -1\quad& 0\\
					1 \quad& 0\quad& 0\\
					0 \quad& 0\quad& 1
				\end{bmatrix}
			$$

			是一个反称矩阵,\textbf{忽略正负号}时,可用$W$表示为,
			$$
				Z = diag(1,1,0)W = diag(1,1,0)W^T
			$$

			得到用$W$表示$[\mathbf{u}]_{\times}$的两种形式(差一个尺度,包括正负号),
			\begin{align}
				[\mathbf{u}]_{\times} &= UZU^T \label{inverse_decompose}\\
				&= U diag(1,1,0)WU^T\\
				&= U diag(1,1,0)W^TU^T
			\end{align}

			注意$[\mathbf{u}]_{\times}$是齐次矩阵,忽略了尺度$k$。

		\item 对任意向量$\mathbf{t}$和非奇异矩阵$\mathbf{M}$,
			\begin{equation}
				[\mathbf{t}]_{\times}\mathbf{M} = \mathbf{M}^{-T}\left[\mathbf{M}^{-1}\mathbf{t}\right]_{\times} \label {inver_m_p}
			\end{equation}
	\end{itemize}