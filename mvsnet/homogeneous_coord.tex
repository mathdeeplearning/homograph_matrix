\section{齐次坐标}
	齐次坐标就是在原坐标后面添加一个维度,值为1,例如2D点$(x,y)$的齐次坐标为$(x,y,z)$,3D点$(x,y,z)$的齐次坐标为$(x,y,z,1)$。\\

	如果已知$(x,y,z,w)$是齐次坐标,转回3D坐标为$(x/w,y/w,z/w)$,即除以最后一维。\\

	基于这个定义,齐次坐标$k(x,y,z,1)$所代表的3D坐标都是相同的,齐次坐标只是关注元素之间的比率,这也是“齐次”的由来。\\

	当我们拿到一个坐标$(x,y,z)$时,需要要弄明白这是3D坐标还是2D坐标的齐次形式。\\

	齐次坐标$(x,y,0)$,对应的2D坐标是什么呢? 考虑$(x,y,\delta), \delta\rightarrow 0$对应的2D坐标,
	$$
		p = \lim_{\delta\rightarrow 0} (x/\delta, y/\delta)
	$$

	当$\delta\rightarrow 0$,$p$点会沿着$(x,y)$方向趋向于无穷,所以$(x,y,0)$为$(x,y)$方向的\textbf{无穷远点}。\\

	通过齐次坐标可以很容易表示无穷远点,那两条平行线是否交于无穷远点呢?

	\subsubsection*{线、面的齐次表示}
		直线$l$,方程:$ax_1 + bx_2 +c =0$,两端乘以任意一个非0因子,都代表同一条直线;所以$(a,b,c)^T$为直线$l$的齐次坐标,点$p$在直线上则表示为$p^Tl = 0$;\\

		平面$\pi$,$ax_1 + bx_2 + cx_3 +d = 0$,也可用齐次坐标$(a,b,c,d)^T$表示,点$p$在面$\pi$上,则$p^T\pi = 0$\\

		两条直线$l,l^\prime$的交点$q$,则可用叉乘来表示$q = l\times l^\prime$\\

		$l=(a,b,c)^T, l^\prime = (a,b,d)^T$,为两条平行直线,其交点为,
		
		\begin{equation}
			l\times l^{\prime} = 
							\begin{vmatrix}
								i & j& k \\
								a & b& c\\
								a & b& d
							\end{vmatrix}
			= (bd-db, ca-ac, 0)^T
		\end{equation}

		的确是一个无穷远点。

	\subsubsection*{虚圆点}		

		平面上圆的方程,
		$$
			(x - a)^2 + (y - b)^2 = r^2
		$$

		用齐次坐标$(x,y,w)^T$对应的欧式坐标表示为,
		$$
			(x - aw)^2 + (y - bw)^2 = r^2w^2
		$$

		这个方程有两个固定解,$(1,i,0)^T, (1,-i,0)^T$,称为\textbf{虚圆点};\\

		\textbf{虚圆点}是所有圆在无穷远复平面上的的交点。\\

		圆和椭圆都是二次方程,两个椭圆有4个交点,两个圆有2个交点,缺的两个交点就是虚圆点。

	\subsubsection*{无穷远线、面}
		经过简单验证可知,任意的无穷远点都在$l_{\infty} = (0,0,1)^T$上,称为\textbf{无穷远线}\\

		所有无穷远直线汇聚成无穷远面$\pi_{\infty} = (0,0,0,1)^T$\\

		射影几何之所以要讨论无穷远元素,是因为这些无穷元素可以变换到图像上的有限元素,只要确定了它们的姿态,整个变换也就确定了。

	\subsubsection*{表示旋转平移}
	 除了表示无穷远点,齐次坐标还可以统一表示旋转和平移,
	 $$
	 	RP +T = \begin{bmatrix}
	 		R \quad& T\\
	 		0 \quad& 1
	 	\end{bmatrix}
	 	\begin{bmatrix}
	 		P\\
	 		1
	 	\end{bmatrix}
	 $$