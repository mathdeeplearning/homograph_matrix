%-*- coding: UTF-8 -*-
\documentclass[hpyerref,UTF8,a4paper,titlepage,12pt,oneside]{ctexbook}
\usepackage{hyperref}
\usepackage{geometry}
\usepackage{xeCJK, fontspec, xunicode, xltxtra,ulem}
\usepackage{amsthm}
\usepackage{amsmath}
\usepackage{amssymb}
\usepackage{mathrsfs}
\usepackage{mathtools}
\usepackage{commath}
\usepackage{listings}
\usepackage{float}
\usepackage{xcolor}
\usepackage{mdframed}

\graphicspath{{images/}}
\geometry{a4paper,bottom=2cm}

\title{MVSNet}
\author{陈国庆}
\date{\today}

\bibliography{plain}

% 定理结构
\theoremstyle{definition}
\newtheorem{definition}{定义}[section]
\newtheorem{theorem}{定理}[section]
\newtheorem{corollary}{推论}[theorem]
\newtheorem{lemma}[theorem]{Lemma}
\renewcommand\qedsymbol{$\blacksquare$}

\begin{document}

\maketitle
\tableofcontents

\section{齐次坐标}
	齐次坐标就是在原坐标后面添加一个维度,值为1,例如2D点$(x,y)$的齐次坐标为$(x,y,z)$,3D点$(x,y,z)$的齐次坐标为$(x,y,z,1)$。\\

	如果已知$(x,y,z,w)$是齐次坐标,转回3D坐标为$(x/w,y/w,z/w)$,即除以最后一维。\\

	基于这个定义,齐次坐标$k(x,y,z,1)$所代表的3D坐标都是相同的,齐次坐标只是关注元素之间的比率,这也是“齐次”的由来。\\

	当我们拿到一个坐标$(x,y,z)$时,需要要弄明白这是3D坐标还是2D坐标的齐次形式。\\

	齐次坐标$(x,y,0)$,对应的2D坐标是什么呢? 考虑$(x,y,\delta), \delta\rightarrow 0$对应的2D坐标,
	$$
		p = \lim_{\delta\rightarrow 0} (x/\delta, y/\delta)
	$$

	当$\delta\rightarrow 0$,$p$点会沿着$(x,y)$方向趋向于无穷,所以$(x,y,0)$为$(x,y)$方向的\textbf{无穷远点}。\\

	通过齐次坐标可以很容易表示无穷远点,那两条平行线是否交于无穷远点呢?

	\subsubsection*{线、面的齐次表示}
		直线$l$,方程:$ax_1 + bx_2 +c =0$,两端乘以任意一个非0因子,都代表同一条直线;所以$(a,b,c)^T$为直线$l$的齐次坐标,点$p$在直线上则表示为$p^Tl = 0$;\\

		平面$\pi$,$ax_1 + bx_2 + cx_3 +d = 0$,也可用齐次坐标$(a,b,c,d)^T$表示,点$p$在面$\pi$上,则$p^T\pi = 0$\\

		两条直线$l,l^\prime$的交点$q$,则可用叉乘来表示$q = l\times l^\prime$\\

		$l=(a,b,c)^T, l^\prime = (a,b,d)^T$,为两条平行直线,其交点为,
		
		\begin{equation}
			l\times l^{\prime} = 
							\begin{vmatrix}
								i & j& k \\
								a & b& c\\
								a & b& d
							\end{vmatrix}
			= (bd-db, ca-ac, 0)^T
		\end{equation}

		的确是一个无穷远点。

	\subsubsection*{虚圆点}		

		平面上圆的方程,
		$$
			(x - a)^2 + (y - b)^2 = r^2
		$$

		用齐次坐标$(x,y,w)^T$对应的欧式坐标表示为,
		$$
			(x - aw)^2 + (y - bw)^2 = r^2w^2
		$$

		这个方程有两个固定解,$(1,i,0)^T, (1,-i,0)^T$,称为\textbf{虚圆点};\\

		\textbf{虚圆点}是所有圆在无穷远复平面上的的交点。\\

		圆和椭圆都是二次方程,两个椭圆有4个交点,两个圆有2个交点,缺的两个交点就是虚圆点。

	\subsubsection*{无穷远线、面}
		经过简单验证可知,任意的无穷远点都在$l_{\infty} = (0,0,1)^T$上,称为\textbf{无穷远线}\\

		所有无穷远直线汇聚成无穷远面$\pi_{\infty} = (0,0,0,1)^T$\\

		射影几何之所以要讨论无穷远元素,是因为这些无穷元素可以变换到图像上的有限元素,只要确定了它们的姿态,整个变换也就确定了。

	\subsubsection*{表示旋转平移}
	 除了表示无穷远点,齐次坐标还可以统一表示旋转和平移,
	 $$
	 	RP +T = \begin{bmatrix}
	 		R \quad& T\\
	 		0 \quad& 1
	 	\end{bmatrix}
	 	\begin{bmatrix}
	 		P\\
	 		1
	 	\end{bmatrix}
	 $$

\section{针孔成像模型}
	针孔模相机成像方式是场景通过相机光心(小孔)在感光元件下留下像,小孔太小会导致成像清晰但暗淡,太大则导致明亮而模糊,对此在小孔前加个凸透镜汇聚光线,使得成像清晰而明亮\footnote{\url{https://en.wikipedia.org/wiki/Pinhole_camera_model}}。

	\begin{figure}[H]
		\begin{center}
			\includegraphics[width=0.48\textwidth]{images/pinhole.png}
			\includegraphics[width=0.48\textwidth]{images/pinhole_coor.png}
		\end{center}
		\caption{成像模型及相机坐标系}
	\end{figure}

	小孔成像又称为透视成像,当我们拍摄平行铁轨时,照片上铁轨不再平行,并且呈近大远小的特点,这是透视成像的典型特点。\\

	成像过程是一个把空间点投影到像素平面的过程:空间点$P=(x,y,z)$,根据平面几何可知,对应像素坐标为$p = (xf/z, yf/z) = f/z(x,y)$($f$为相机焦距)。\\
	
	$z$不是常数,所以这不是一个线性变换。如果场景点都位于一张平面上,或者场景的深度远远小于场景到相机的距离,可近似认为所有场景点深度都为$z_0$,此时为线性变换,
	$$
		p = \frac{f}{z_0}(x,y)
	$$

	这样的相机称也为\textbf{弱投影相机}或\textbf{仿射相机}。

\subsection{坐标系}
	一般我们要使用到3个坐标系,

	\subsubsection*{世界坐标系}
		世界坐标系用来测量场景的绝对位置,比如可以把原点设置在故宫大殿的中心,这样所有的场景点都会存在一个绝对坐标。\\

		实际上世界坐标只是为了区分不同相机的姿态,无需指定一个绝对位置,常选择第一个相机为世界坐标系。

	\subsubsection*{相机坐标系}
		用来刻画场景相对相机光心的姿态,以光心为原点,光轴为$z$轴,以指向场景的方向为正向;\\

		$xy$平面平行于成像平面,上为$y$轴,右为$x$轴,$z-y-x$符合右手法则;\\

		不同的相机对相同场景,会在各自坐标系中刻画出不同的姿态,在世界坐标系中,可以明确相机坐标系之间的转换关系,统一场景的唯一表示。

	\subsubsection*{像素坐标系}
		像素坐标系$uv$就是2D成像平面,单位是像素,平行于相机坐标系的$xy$平面。\\

		$uv$坐标系的原点一般在图像的左下角或左上角,而$xy$平面的原点在光心,也即$uv$平面的中心,在像素坐标系光心存在一个偏移量$c_x,c_y$(单位为像素)。\\

		世界坐标系和相机坐标系的单位是m,而像素坐标系的单位是pixel,在投影的时候需要明确转换因子,即当前相机1m等于多少pixel

\subsection{相机内参矩阵}
	在相机坐标系中,场景成像与相机本身参数有关,这些参数称为\textbf{内参数},

	\begin{itemize}
		\item 相机焦距$f$,可以调节成像平面与光心的距离,称为\textbf{物理变焦};焦距大成像大,焦距小成像小
		\item 光心偏移量$c_x,c_y$(单位为像素)
		\item 相机偏斜,因为工艺问题,感光元件构成的像平面不是矩形,存在一个角度$\theta$
		\item $x,y$方向可能存在不同的像素转换因子
	\end{itemize}

	这些参数都考虑进来,得到相机\textbf{内参矩阵} $K$,

	\begin{equation}
		K = \begin{bmatrix}
			\alpha \quad& -\alpha\cot\theta        \quad& c_x\\
			0      \quad& \frac{\beta}{\sin\theta} \quad& c_y\\
			0      \quad& 0                        \quad& 1
		\end{bmatrix}
	\end{equation}

	\textit{焦距和转换因子已经融合到$\alpha,\beta$参数中。}\\

	相机坐标系点$P$,在像素平面的投影为,

	$$
		P^{\prime}  = KP
	$$

	上面得到的$P^{\prime}$是2D平面的齐次坐标,转换为2D坐标还需,
	$$
		p = \left(\frac{K_1P}{K_3P}, \frac{K_2P}{K_3P}\right)^T
	$$
	$K_1,K_2,K_3$是$K$的行向量。\\

	对任意尺度因子$d$($d\neq 0$),$dP$为经过光心和$P$的射线,投影像素坐标为,
	$$
		p_d = p
	$$

	这说明射线上所有的点都投影到相同像素上,相机投影实际就是一个降维的过程。\\

	反之,已知像素平面上一点$p$(齐次坐标),反投影到3D空间会得到一条射线,

	\begin{equation}
		P(d) = dK^{-1}p \label{inverse_proj}
	\end{equation}

\subsection{相机投影矩阵}
	内参矩阵刻画场景在相机坐标系下的投影关系,如果场景在世界坐标系下,需要把场景从世界坐标系通过旋转平移变换到相机坐标系,再用内参矩阵进行投影。
	$$
		M = K\left[R\quad T\right]
	$$

	$R,T$为相机坐标系的旋转矩阵和平移向量,$\left[R\quad T\right]$称为相机\textbf{外参矩阵},$M$称为相机的\textbf{投影矩阵}。\\

	投影矩阵是内参矩阵乘以外参矩阵,是一个$3\times 4$的矩阵。\\

	也可以对投影矩阵增广到$4\times 4$的矩阵$\tilde{M}$,

	$$
		\tilde{M} = \begin{bmatrix}
			KR\quad& KT\\
			0\quad& 1
		\end{bmatrix}
	$$

	$\tilde{M}P$投影得到一个$4\times 1$的向量,在代数上这个向量前3维代表像素的齐次坐标,但在几何上没有意义。\\

	增广矩阵主要是能在反投影时简化计算,后面会用到。

\subsection{3D空间的线性变换}

	在MVS任务中要着重研究各个相机成像平面之间的变换关系,下面是3D空间到自身的一些变换(2D空间也类似)。\\

	变换的本质是探究在某些变换群下,那些不变的性质,所以下面每个变换都探讨什么不变,所用的都是其次坐标。

	\subsubsection{欧式变换}
		对目标进行平移和旋转,保持目标大小、长度、夹角都不变。表示矩阵为,
		$$
			H_E = \begin{bmatrix}
				\mathbf{R}_{3\times 3}\quad& \mathbf{t}_{3\times 1}\\
				0_{1\times 3} \quad& 1_{1\times 1}
			\end{bmatrix}
		$$

		$R$是$3\times 3$的旋转矩阵,$t$是平移向量。\\

		可以验证虚圆点是$H_E$的两个特征向量。这说明欧式变换的本质是虚圆点不变,大小、长度、夹角都是因为虚圆点不变而成立。\\

		\textit{2D欧式变换有3个自由度,3D的有6个自由度}

	\subsubsection{相似变换}
		均匀缩放的欧式变换,目标大小、长度发生变化,但夹角不变,
		$$
			H_S = \begin{bmatrix}
				s\mathbf{R}_{3\times 3}\quad& \mathbf{t}_{3\times 1}\\
				0_{1\times 3} \quad& 1_{1\times 1}
			\end{bmatrix}
		$$

		$s$是缩放尺度,3个方向上相同。可以验证,虚圆点也是相似变换的特征向量。\\

		\textit{2D相似变换有4个自由度,3D的有7个自由度}
	
	\subsubsection{仿射变换}
		非均匀缩放的欧式变换,夹角不能保持,但能保持平行线变换后仍为平行,无穷远平面不变。
		$$
			H_A = \begin{bmatrix}
				\mathbf{A}_{3\times 3}\quad& \mathbf{t}_{3\times 1}\\
				0_{1\times 3} \quad& 1_{1\times 1}
			\end{bmatrix}
		$$

		$A$是一个可逆矩阵。\\

		标准无穷远平面$\pi_{\infty} = (0,0,0,1)^T$,根据点面对偶原理,平面变换为,
		\begin{align*}
			\pi_{\infty}^{\prime} = H_A^{-T}\pi_{\infty} 
			&=
			\begin{bmatrix}
				\mathbf{A}_{3\times 3}\quad& \mathbf{t}_{3\times 1}\\
				0_{1\times 3} \quad& 1_{1\times 1}
			\end{bmatrix}^{-T}
			\begin{bmatrix}
				\mathbf{0}_{3\times 1}\\
				1
			\end{bmatrix}\\
			&=
			\begin{bmatrix}
				\mathbf{A}^{-T}\quad& \mathbf{0}\\
				-\mathbf{t}^TA^{-T} \quad& 1
			\end{bmatrix}
			\begin{bmatrix}
				\mathbf{0}\\
				1
			\end{bmatrix}\\
			&= (0,0,0,1)^T\\
			&=\pi_{\infty} 
		\end{align*}

		容易验证,$l_{\infty}, \pi_{\infty}$是仿射映射的特征向量。\\

		仿射变换可分解为(以2D为例),
		$$
			\mathbf{A} = \mathbf{R}(\theta)\mathbf{R}(-\phi)
			\begin{pmatrix}
				s_x \quad &0\\
				0 \quad & s_y
			\end{pmatrix}
			\mathbf{R}(\phi)
		$$

		$s_x,s_y$为在$x,y$方向的放缩因子,$\theta,\phi$为旋转角度,如此可以看出仿射变换的确是非均匀缩放的欧式变换。\\

		\textit{2D仿射变换有6个自由度,3D的有12个自由度}
	
	\subsubsection{射影变换}

		\textbf{射影变换} 又称为 \textbf{单应变换},这两个名字会混着用。\\

		对非奇异$\mathbf{H}$,变换$x^\prime = \mathbf{H}x + \mathbf{b}$;若$\mathbf{b} = 0$,则为射影变换;否则不是。\\

		变换矩阵具有更一般的形式,平行性和无穷远平面也保持不了,但能保持线性性,即直线仍映射为直线。
		$$
			H_P = \begin{bmatrix}
				\mathbf{A}_{3\times 3}\quad& \mathbf{t}_{3\times 1}\\
				\mathbf{v}_{1\times 3} \quad& 1_{1\times 1}
			\end{bmatrix}
		$$

		以2D为例,对直行$x^Tl = 0$,变换后的点$x^\prime = H_Px$,变换后的$l^\prime = H_P^{-T}l$,

		$$
			x^\prime l^\prime = \left(H_Px\right)^TH_P^{-T}l = x^T H_P^TH_P^{-T}l = 0
		$$

		2D空间中的确直线变为直线。3D空间的直线表示比较复杂,需用两个平面交线表示:$x^T\pi_A =0,  x^T\pi_B =0$,
		\begin{align*}		
			x^\prime \pi_A^{\prime} &= (H_Px)^T(H_P^{-T}\pi_A) = x^TH_P^TH_P^{-T}\pi_A = 0\\
			x^\prime \pi_B^{\prime} &= (H_Px)^T(H_P^{-T}\pi_B) = x^TH_P^TH_P^{-T}\pi_B = 0
		\end{align*}

		所以3D空间的线性性也得到保证。如下情况会存在平面间的射影变换,

		\begin{itemize}
			\item 场景如果是平面,则场景平面与像平面间的变换
			\item 场景是平面,则不同相机的成像平面之间的变换
			\item 同一相机,只有旋转没有平移拍摄到的像平面之间的变换;这种拍摄方式常用来做广角图拼接
			\item 同一光心透视两张平面之间形成的变换
		\end{itemize}

		如果场景不是平面,则不同相机像平面之间一般不是射影变换,通过极几何约束这种对应关系。\\

		射影变换可以进行层次化分解,$\mathbf{H} = \mathbf{H_S}\mathbf{H_A}\mathbf{H_P}$,
		$$
			\begin{pmatrix}
				\mathbf{A}_{3\times 3}\quad& \mathbf{t}_{3\times 1}\\
				\mathbf{v}_{1\times 3} \quad& 1_{1\times 1}
			\end{pmatrix}
			=
			\begin{pmatrix}
				s\mathbf{R}_{3\times 3}\quad & \mathbf{t}_{3\times 1}\\
				\mathbf{0}_{1\times 3}\quad & 1_{1\times 1}
			\end{pmatrix}
			\begin{pmatrix}
				\mathbf{K}_{3\times 3}\quad & \mathbf{0}_{3\times 1}\\
				\mathbf{0}_{1\times 3}\quad & 1_{1\times 1}
			\end{pmatrix}
			\begin{pmatrix}
				\mathbf{I}_{3\times 3}\quad & \mathbf{0}_{3\times 1}\\
				\mathbf{v}_{1\times 3}\quad & 1_{1\times 1}
			\end{pmatrix}
		$$
		\textit{2D透视变换有7个自由度,3D的有15个自由度}
	\subsubsection{仿射恢复}

		仿射变换使得无穷远点保持不变,而射影变换则使得无穷远点变为有限点,如下图\footnote{\url{https://www.graphicsmill.com/docs/gm/affine-and-projective-transformations.htm}},

		\begin{figure}[H]
			% \begin{center}
			\begin{minipage}[t]{0.52\linewidth}
				\centering
				\includegraphics[width=0.8\textwidth]{images/origin_.jpeg}
				\caption{墙面原图}
			\end{minipage}
			\begin{minipage}[t]{0.78\linewidth}
				\centering
				\includegraphics[width=0.8\textwidth]{images/projective_.png}
				\caption{射影变换,无穷远点被映射为有限点}
			\end{minipage}			
		\end{figure}

		可以把无穷远点的像$p^{\prime}_{\infty}$映射回标准位置$(0,0,1)$,来恢复图像的仿射性质,具体来说,

		$$
			\mathbf{H}\mathbf{H_P}^{-1} = \mathbf{H_S}\mathbf{H_A}
		$$

		只要在图像上测量$p^{\prime}_{\infty}$的像素坐标,便可纠正图像的射影失真,恢复仿射性质,

		\begin{figure}[H]
			\begin{minipage}[t]{0.52\linewidth}
				\centering
				\includegraphics[width=0.8\textwidth]{images/origin_.jpeg}
				\caption{墙面原图}
			\end{minipage}		
			\begin{minipage}[t]{0.5\linewidth}
				\centering
				\includegraphics[width=0.8\textwidth]{images/affine_.jpeg}
				\caption{仿射恢复,通过无穷远点消除射影失真}
			\end{minipage}			
		\end{figure}
		
		如果观察到更多的条件,可以继续消除仿射失真,恢复欧式性质。这就是变换的层次分解带来的好处。

\subsection{反称矩阵}
 	向量$\mathbf{u},\mathbf{v}$的叉乘,
 	\begin{align*}
 		\mathbf{u}\times \mathbf{v} &= 
 		\begin{vmatrix}
 			\mathbf{i}\quad & \mathbf{j}\quad & \mathbf{k}\\
 			u_1\quad& u_2\quad& u_3\\
 			v_1\quad& v_2\quad& v_3
 		\end{vmatrix}\\
 		&= (u_2v_3 -u_3v_2)\mathbf{i} + (u_3v_1 - u_1v_3)\mathbf{j} +(u_1v_2 - u_2v_1)\mathbf{k}\\
 		&=\begin{bmatrix}
 			0 \quad& -u_3\quad & u_2\\
 			u_3\quad & 0\quad& -u_1\\
 			-u_2\quad& u_1\quad& 0
 		\end{bmatrix}
 		\begin{bmatrix}
 			v_1\\
 			v_2\\
 			v_3
 		\end{bmatrix}
 		\\
 		&= [\mathbf{u}]_{\times} \cdot \mathbf{v}
 	\end{align*}

 	$[\mathbf{u}]_{\times}$为由$\mathbf{u}$张成的\textbf{反称矩阵},
 	$$
 		[\mathbf{u}]_{\times} = 
 		\begin{bmatrix}
 			0 \quad& -u_3\quad & u_2\\
 			u_3\quad & 0\quad& -u_1\\
 			-u_2\quad& u_1\quad& 0
 		\end{bmatrix}
 	$$

	存在如下性质,

	\begin{itemize}
		\item 对任意向量$p$,$p^T [\mathbf{u}]_{\times} p = 0$
		\item 矩阵秩为2,特征值为和纯虚数;所以基础矩阵秩为2
		\item 二次幂可展开为,
			$$
				[\mathbf{u}]_{\times}^2 = \mathbf{u}\mathbf{u}^T - \Vert\mathbf{u}\Vert^2 \mathbf{I_3}
			$$
		\item 三次幂等价于自身,
			$$
				[\mathbf{u}]_{\times}^3 = -\Vert\mathbf{u}\Vert^2[\mathbf{u}]_{\times}
			$$
		\item 正交分解,
			$$
				[\mathbf{u}]_{\times} = kUZU^T
			$$

			$U$是单位正交矩阵,$k$是一个尺度因子,$\mathbf{u}$如果是齐次坐标,可以忽略;$Z$为,
			$$
				Z = \begin{bmatrix}
					0 \quad& 1\quad& 0\\
					-1 \quad& 0\quad& 0\\
					0 \quad& 0\quad& 0
				\end{bmatrix}
				\quad
				 W = \begin{bmatrix}
					0 \quad& -1\quad& 0\\
					1 \quad& 0\quad& 0\\
					0 \quad& 0\quad& 1
				\end{bmatrix}
			$$

			是一个反称矩阵,\textbf{忽略正负号}时,可用$W$表示为,
			$$
				Z = diag(1,1,0)W = diag(1,1,0)W^T
			$$

			得到用$W$表示$[\mathbf{u}]_{\times}$的两种形式(差一个尺度,包括正负号),
			\begin{align}
				[\mathbf{u}]_{\times} &= UZU^T \label{inverse_decompose}\\
				&= U diag(1,1,0)WU^T\\
				&= U diag(1,1,0)W^TU^T
			\end{align}

			注意$[\mathbf{u}]_{\times}$是齐次矩阵,忽略了尺度$k$。

		\item 对任意向量$\mathbf{t}$和非奇异矩阵$\mathbf{M}$,
			\begin{equation}
				[\mathbf{t}]_{\times}\mathbf{M} = \mathbf{M}^{-T}\left[\mathbf{M}^{-1}\mathbf{t}\right]_{\times} \label {inver_m_p}
			\end{equation}
	\end{itemize}

\section{对极约束}
	如前所述,对一般场景而言,两个成像平面构不成射影变换,它们服从更一般的\textbf{极几何}约束,可用一个矩阵$F$表示这种约束关系,矩阵$F$称为\textbf{基础矩阵},
	\begin{figure}[H]
		\begin{center}
			\includegraphics[width=0.8\textwidth]{../images/base_matrix.png}
		\end{center}
		\caption{两图像之间的极几何约束,$O_1,O_2$为相机\textbf{光心},$O_1O_2$为\textbf{基线},$p,p^{\prime}$为$P$在两个像素平面的像,$e,e^\prime$为\textbf{极点},$l,l^{\prime}$为\textbf{极线}}
	\end{figure}
	\begin{itemize}
		\item 世界坐标系选为第一个相机坐标系
		\item $K,K^{\prime}$是两个相机的内参矩阵
		\item $R,T$是第二个相机相对第一个相机的旋转和平移
		\item $[T]_{\times}$是$T$张成的反称矩阵,秩为$2$
		\item $e,e^\prime$为极点,$e^\prime$是$e$的投影,存在
			$$
				e^\prime =K^\prime T
			$$	
	\end{itemize}

	所有极线都过极点,所以,
	$$
		{p^\prime}^T l^\prime = 0, \quad {e^\prime}^T l^\prime = 0
	$$
	\subsection*{基础矩阵}
		$p,p^\prime$是同一个空间点$P$在两个像素平面的投影,将$p$反投影到$P$,再投影到第二个像素平面可得到$p^\prime$,根据(\ref{inverse_proj}),

		\begin{align}
			p^{\prime}_d &= K^{\prime}[R\quad T]
			\begin{bmatrix}
				dK^{-1}p\\
				1
			\end{bmatrix}\nonumber\\
			&= K^{\prime}\left(dRK^{-1}p + T\right)\nonumber\\
			&= dK^{\prime}RK^{-1}p + K^{\prime}T\label{f_inverse}\\
			&= dK^{\prime}RK^{-1}p + e^\prime\label{f_inverse_e}
		\end{align}

		尺度因子$d$在齐次坐标下并无影响,只是提醒反投影后存在一个尺度。\\

		由此可见,一般情况下$p,p^\prime$之间并非射影变换。\\

		(\ref{f_inverse_e})式两边叉乘$e^\prime$,

		$$
			p^{\prime}_d\times e^\prime = dK^{\prime}RK^{-1}p \times e^\prime
		$$

		转置一下,
		$$
			e^\prime \times p^{\prime}_d = e^\prime \times dK^{\prime}RK^{-1}p = d[e^\prime]_{\times}K^{\prime}RK^{-1}p
		$$

		两边与$p^\prime_d$作内积,
		$$
			p^\prime_d [e^\prime]_{\times}K^{\prime}RK^{-1}p = 0
		$$

		称,
		\begin{equation*}
			\mathbf{F} = [e^\prime]_{\times}K^{\prime}RK^{-1}
		\end{equation*}

		为\textbf{基础矩阵},任意点对$p^\prime,p$都满足约束,

		\begin{equation}
			{p^{\prime}}^T \mathbf{F} p = 0 \label{f_constrain}
		\end{equation}

		根据(\ref{inver_m_p}),可得到$F$的另一表达式,

		$$
			\mathbf{F} = [e^\prime]_{\times}K^{\prime}RK^{-1} = {K^{\prime}}^{-T}[T]_{\times}RK^{-1}
		$$

		下面两个表达式都是$F$的常用形式,
		\begin{align}
			\mathbf{F} &= [e^\prime]_{\times}K^{\prime}RK^{-1} \label{f_1}\\
			\mathbf{F} &= {K^{\prime}}^{-T}[T]_{\times}RK^{-1} \label{f_1}
		\end{align}

		$[T]_{\times},[e^\prime]_{\times}$的秩为2,所以$F$的秩也为2。\\

	\subsection*{点线对应}

		$p^{\prime}$在极线$l^{\prime}$上,而$p^\prime \mathbf{F} p=0$,可知,

	\begin{align*}
		l^{\prime} = Fp,\quad 
		l = F^Tp^\prime
	\end{align*}

	基础矩阵描述了点之间的约束关系,每个点对应一条极线。\\

	极点$e^\prime$也在极线$l^\prime$上,
	$$
		{e^\prime}^T l^\prime = 0 \Rightarrow {e^\prime}^TFp=0 \Rightarrow p^T F^T e^\prime = 0
	$$

	因为$p$的任意性,可知

	$$
		{e^\prime}^T F = 0, \quad Fe = 0
	$$

	沿着极线$l^\prime$搜索$p$的像,会大幅缩小搜索范围。

	\subsection*{工程实现}
		在实际计算时,通过投影矩阵的增广的投影矩阵表示更为方便,
		\begin{equation}
			N_d= M^{\prime}M^{-1} = \begin{bmatrix}
				dK^\prime R K^{-1} \quad& K^\prime T\\
				0\quad& 1\quad
			\end{bmatrix}\label{extend_f}		
		\end{equation}

		从$N_d$中截取出左上角$3\times 3$的矩阵便是旋转矩阵,右上角$3\times 1$的向量是平移向量,这在各种工具中非常容易实现。

\section{外参恢复}\label{section_recovery_outer_p}
	基础矩阵$F$包含了相机的内外参数,如果相机内参数和$F$已知,能否从$F$中分离出外参数?\\

	$$
		\mathbf{F} = {K^{\prime}}^{-T}[T]_{\times}RK^{-1} \Rightarrow {K^{\prime}}^{T}\mathbf{F} K = [T]_{\times}R = \mathbf{E}
	$$

	其中,
	$$
		\mathbf{E} = [T]_{\times}R
	$$

	称为\textbf{本质矩阵}。现在问题简化为如何从$E$中分离出外参$R,T$。\\

	根据(\ref{inverse_decompose}),$[T]_{\times}$可分解为,

	\begin{align*}
		[T]_{\times} &= U diag(1,1,0)WU^T\\
		[T]_{\times} &= U diag(1,1,0)W^TU^T
	\end{align*}

	这两种分解仅差一个尺度或者符号。$E$可表示为,

	\begin{align*}
		\mathbf{E} = U diag(1,1,0)\left(WU^TR\right)\quad 
		\text{or} \quad 
		U diag(1,1,0)\left(W^TU^TR\right)
	\end{align*}

	注意$E$是秩为2的反称矩阵,奇异值分解形式为,

	$$
		\mathbf{E} = U diag(1,1,0) V^T
	$$

	在$\mathbf{E} $确定的情况下,$U,V$是已知量,对比可知,
	\begin{align*}
		V^T = WU^TR \quad \text{or}\quad W^TU^TR
	\end{align*}

	得到$R$的两种表达式,
	\begin{align*}
		R = UW^TV^T\quad \text{or}\quad UWV^T
	\end{align*}

	$R$是旋转矩阵,所以行列式值为正,修正一下符号,

	$$
		R \leftarrow (\mathop{det} R)R
	$$

	而,
	$$
		[T]_{\times} = R^T\mathbf{E}
	$$

	这样得到$T$的反称矩阵,可以组合出$T$向量;$R$有两个值,$T$也对应有两个值。\\


	这两组解几何表示,一组场景都在相机前面;一组场景在相机后面,通过重建出的点过滤掉在后面的解即可。

\section{单应变换}

	如果拍摄场景是一张平面,法向量为$\mathbf{n}$,到原点距离为$d$,则平面方程为,
	$$
		\mathbf{n}^T P = d
	$$

	$\mathbf{n}_d = \mathbf{n}/d$,场景平面可表示为,
	
	$$
		\mathbf{n}_d^T P = 1
	$$

	像素点$p$反投影为$dK^{-1}p$,

	\begin{align*}
		p^{\prime}(d) &= K^{\prime}\left(dRK^{-1}p + T\right)\\
		&= K^{\prime}\left(dRK^{-1}p + T\mathbf{n}_d^T P\right)\\
		&= K^{\prime}\left(dRK^{-1}p + T\mathbf{n}_d^TdK^{-1}p\right)\\
		&= dK^{\prime}\left(R + T\mathbf{n}_d^T\right)K^{-1}p\\
		&= Hp
	\end{align*}

	这里的关键是第二步,代入场景平面方程,把$p$给分离了出来,$p^\prime,p$之间构成射影变换。\\

	$K^\prime$是齐次矩阵,与$dK^\prime$等价。

	\begin{equation}
		H= K^{\prime}\left(R + T\mathbf{n}_d^T\right)K^{-1} \label{homograph_matrix}
	\end{equation}

	称为\textbf{单应矩阵},因此射影变换也称为\textbf{单应变换}。\\

	$p,p^{\prime}$依然服从基础矩阵的约束,(\ref{extend_f})式的增广表示也包含了单应变换这一特殊情况。\\

	基础矩阵只是描述点与极线的对应关系,而单应矩阵描述点之间一一对应关系,这也是“\textbf{单应}”的意义。

	\subsection*{论文中的公式}

	在MVSNet中,只知道两个相机在世界坐标系的旋转和平移$R_1,t_1,R_2, t_2$,相对旋转$R$和平移$t$为,

	$$
		R = R_2R_1^{-1},\quad 
		t = t_2 - R_2R_1^{-1}t_1
	$$

	(\ref{homograph_matrix})用世界坐标系可表示为,

	\begin{align}
		H &= K_2 \left(R_2R_1^{-1} +\left(t_2 - R_2R_1^{-1}t_1\right) \mathbf{n}_d^T\right) K_1^{-1} \label{new_homograph_matrix}
	\end{align}	

	原论文中的公式是错误的,这是正确版本。\\

	增广表示蕴含了单应变换,这个复杂的式子在编程中并不会被使用到。
\section{SfM}
	\definition{SfM问题} 也称为\textbf{欧式结构恢复问题},已知$m$张图片和对应的相机内参矩阵$K_i(i=1,\cdots,m)$,求解:
	\begin{itemize}
		\item \textbf{Structure}:$n$个三维点$X_j(j=1,\cdots\,n)$的坐标
		\item \textbf{Motion}:$m$个相机的外参数$R_i,T_i(i=1,\cdots,m)$
	\end{itemize}		
	
	如果没有额外信息,无法恢复出场景的绝对坐标(经纬度)、朝向及尺度,恢复的最好结果是与真实场景之间差一个相似变换;\\

	比如,如果知道场景中某两点之间的真实距离,则可恢复出场景的尺度,但依然无法确定绝对坐标和朝向。

	\subsection{平行视图}
		在传统两视图(\ref{two_stereo})重建场景,两个相机只有平移而无旋转,此时$R = \mathbf{I}$,基础矩阵可简化为,
		$$
			\mathbf{F} = [e^\prime]_{\times}K^{\prime}K^{-1}
		$$

		一般两个相机会采用相同的参数,此时$K = K^\prime$,基础矩阵可进一步简化为,
		$$
			\mathbf{F} = [e^\prime]_{\times}
		$$

		平行视图的场景,极点为无穷远点,极线都平行于基线,此时$e^\prime = (1,0,0)^T$,

		$$
			\mathbf{F} =  \begin{bmatrix}
				0\quad & 0\quad& 0\\
				0\quad & 0\quad& -1\\
				0\quad & 1\quad& 0
			\end{bmatrix}
		$$

		根据(\ref{f_inverse_e}),
		\begin{equation}
			p^\prime_d = dp + e^\prime = d(u+1/d,v,1)^T \label{disparity_pair}
		\end{equation}
		$$
			p^\prime_d = dp + e^\prime = d(u+1/d,v,1)^T
		$$

		此时,$p^\prime_u = u + 1/d, p^\prime_v = v$,即对应点的$v^\prime$坐标与原$v$坐标坐标是相同的,两条极线具有相同的高度,此时极线也称为\textbf{扫描线}。

		\begin{figure}[H]
			% \begin{center}
			\begin{minipage}[t]{0.49\linewidth}
				\centering
				\includegraphics[width=\textwidth]{images/two_stereo.png}
				\caption{平行视图}
				\label{two_stereo}
			\end{minipage}
			\begin{minipage}[t]{0.49\linewidth}
				\centering
				\includegraphics[width=\textwidth]{images/two_stereo_depth.png}
				\caption{平行视图俯视几何}
				\label{two_stereo_gemotry}
			\end{minipage}			
		\end{figure}

		沿着扫描线,平行视图的深度计算非常容易,如图(\ref{two_stereo_gemotry})两个三角形相似,可得,
		$$
			p_u - p^\prime_u = \frac{Bf}{z}
		$$

		$p_u - p^\prime_u$称为\textbf{视差},是同一3D点在两个成像平面的水平像素差。$B,f$均为已知量,以此知视差与深度成反比。

		\subsubsection*{极线校正}
			上面说的平行视图是非常理想的情况,实际上之前提到过,因为工艺的原因,很难保证单相机的像平面是矩形,也很难保证两个相机的像平面是绝对平移的;\\

			对此需要将两个像平面修正为平行平面,此时极线才能真正构成\textbf{扫描线},这个过程称为\textbf{极线校正}。具体来说,先把第二个平面的极点$e^\prime$变换到$(1,0,0)^T$,得到变换矩阵$H^\prime$;再利用投影误差最小化,得到第一个平面的变换矩阵$H$,如此可使得两极线对齐。具体细节请参考鲁鹏老师课程。
			\begin{figure}[H]
				\begin{center}
					\includegraphics[width=0.8\textwidth]{images/epipolar.png}
				\end{center}
				\caption{极线校正}
			\end{figure}

		\subsubsection*{Matching Cost}
			根据(\ref{disparity_pair}),

			$$
				p_u - p^\prime_u = \frac{1}{d}
			$$

			视差与反投影距离$d$成反比,针对某个具体的点$p$,此处的$d$是某个确定的值$d_0$;\\

			不同的$d$会产生不同的视差,给$d$一个范围,逐一计算视差范围内的代价,每一代价$c_i$对应$p_i^\prime$。\\

			$w\times h$的图片,会产生$w\times h\times d$的的张量,称为\textbf{匹配代价},MVSNet的的\textbf{Cost Volume}实际是将这一概念从两视图推广到了多视图。\\

			\begin{figure}[H]
				\begin{center}
					\includegraphics[width=0.85\textwidth]{images/matching_cost.jpeg}
				\end{center}
				\caption{Matching Cost \& 二次曲线修正\protect\footnote{\url{https://blog.csdn.net/rs_lys/article/details/83302323}}}
			\end{figure}

			那些平行于像平面的场景点深度是相同的,产生的视差也是相同的,这样的平面称也为\textbf{fronto-parallel}平面;\\

			而场景中的斜面(\textbf{slanted plane}),虽然也是平面,但上面点的深度是不同的,所以产生的视差也是不同的。\\

			在计算匹配代价时,像素视差误差会平方放大深度误差($d$为视差,$D$为深度),

			$$
				\Delta d= -\frac{Bf}{D^2}\Delta D \Rightarrow \Delta D = -\frac{D^2}{Bf}\Delta d
			$$
			
			所以需要通过视差变化来刻画点对之间的匹配程度,比如,视差变化很小的区域可能处在同一\textbf{fronto-parallel}平面;视差变化连续的区域可能处在同一\textbf{slanted plane}上;而视差变化剧烈的区域可能处在不同的曲面或曲率很大的曲面上。\\

			这是各种文献里都要对fronto-parallel plane、slanted plane及其他情况进行区分的原因。

		\subsubsection*{双目立体匹配}
			双目立体成本较低,一般用在自动驾驶或机器人场景,传统的匹配过程包括四个步骤:

			\begin{enumerate}
				\item 代价计算,具体就是衡量在两张图像上点对的匹配代价,代价可以是两点灰度绝对值差、绝对值和、归一化相关系数、互信息、Census变换、Rank变换等方法衡量;对匹配范围而已,主要是局部计算,可以点点计算,也可邻域匹配计算
				\item 代价聚合,通过局部或者全局的方式进一步调优上一步的匹配代价
					\begin{itemize}
						\item 局部方法,通过\textbf{支撑窗口}做均值滤波,又分为固定窗和自适应窗
						\item 全局方法,定义一个能量函数,最小化整体点点代价匹配,转化为一个优化问题
					\end{itemize}
				\item 视差计算,取代价最小的视差作为真值
				\item 视差优化,手段包括提高精度,比如上图通过二次曲线拟合将视差整数值扩展到浮点值;剔除错误匹配、弱纹理优化、填补空洞等
			\end{enumerate}

			最为提及的概念便是局部方法中的\textbf{support window},实际就是一个常见的邻域均值,以此抑制单点噪音。\\

			但窗内的像素可能因深度不同而导致视差不同,固定窗的均值滤波可能会因此引入视差噪音,从而放大深度误差,因此有一些自适应窗的方法来应对这个问题,这里不再展开。\\

			实际上,全局方法和局部方法一样,也暗含了邻域视差不变的假设\footnote{\url{http://vision.deis.unibo.it/~smatt/Seminars/StereoVision.pdf}}。


	\subsection{两视图重建}

	考虑两张图片,只要能估计出基础矩阵$F$,便可根据\ref{section_recovery_outer_p}节估计出$R,T$,
	$$
		p^\prime F p = 0
	$$

	根据基础矩阵约束,$p=(u,v,1)^T,p^{\prime} = (u^{\prime}, v^{\prime},1)$
	$$
		(u^{\prime}, v^{\prime},1)
		\begin{bmatrix}
			F_{11}\quad& F_{12}\quad& F_{13}\\
			F_{21}\quad& F_{22}\quad& F_{23}\\
			F_{31}\quad& F_{32}\quad& F_{33}\\
		\end{bmatrix}
		(u,v,1)^T = 0
	$$
	展开可得,

	$$
		\left(uu^{\prime}, vu^{\prime}, u^{\prime}, uv^{\prime},vv^{\prime},v^{\prime},u,v,1\right)
		\begin{bmatrix*}
			F_{11}\\
			F_{12}\\
			F_{13}\\
			F_{21}\\
			F_{22}\\
			F_{23}\\
			F_{31}\\
			F_{31}\\
			F_{33}
		\end{bmatrix*} = 0
	$$

	在不考虑尺度的情况下,$F$矩阵有8个未知数,因此至少需要8对点才能确定$F$,实际上会用大于8对点,求一个最小二乘解。\\

	$F$的秩为2,最小二乘解得到的矩阵秩一般为3,需要做一次SVD分解,置最小特征值为0,来得到$F$。

	\subsubsection*{解的存在性}
		并不是任意8对点都能求解出$F$,如果8个方程中部分线性相关,会导致方程的数量少于未知数的数量,从而没有唯一解。\\

		如果场景是一张平面,$P_i(i=1,2,3)$是平面上不共线的三点,对应像素坐标$p_i,p^\prime_i$;

		平面上任意点$P$可表示为$P_i$的线性组合,$P=\alpha P_1 + \beta P_2 + \gamma P_3$,$P$的投影为,
		
		\begin{align*}
			p &= K\left(\alpha P_1 + \beta P_2 + \gamma P_3\right)\\
				&= \alpha KP_1 + \beta K P_2 + \gamma KP_3\\
				&= p_1 + p_2 + p_3
		\end{align*}

		场景平面存在单应变换,
		$$
			p^\prime = Hp = Hp_1 + Hp_2 + Hp_3 = p^\prime_1+p^\prime_2+p^\prime_3
		$$

		因此,点对$(p_1 + p_2 + p_3,p^\prime_1+p^\prime_2+p^\prime_3)$贡献了一个冗余方程,此时基础矩阵没有唯一解,实际上要使得线性无关的方程数量至少为8,方程才有唯一解。\\

		在单应变换的情况下,需求解单应矩阵,单应矩阵包含了基础矩阵所有的元素,以此来构造基础矩阵。\\

		为了避免陷入讨论解的存在性,在重建场景时,经常用基础矩阵和单应矩阵分别求解一次,哪个效果好用哪个。

	\subsubsection*{确定对应点对}
		怎么确定两张图像的对应点对$p,p^{\prime}$?可以用传统的SIFT特征匹配,也可以用深度学习的方法,这里就不再详细介绍。

	
	注意这里世界坐标系选择为第一个相机的坐标系,因此只有第二个相机的$R,T$需要求解,得到基础矩阵$R,T$后,根据投影关系,
	$$
		P_d = dK^{-1}p
	$$

	可计算出空间点$P$的3D坐标,也称点云,这就是\textbf{三角化}方法。\\

	这个坐标是在第一个相机坐标系下的坐标,并且重建出的目标会跟真实大小相差一个尺度$d$。

	\subsection{多视图重建}
		多张图片,多个相机内参矩阵,求解SfM问题常用\textbf{增量法},具体来说包括几个步骤,
	
	\begin{itemize}
		\item 相机之间通过两视图方法两两求解$F_{ij}$
		\item 从一对相机开始,不断加入其他相机,这样重建的点会越来越多
		\item 持续加入相机会累积误差,所以通过\textbf{捆绑调整}(Bundle Adjustment)来做一个整体优化
	\end{itemize}

	其中细节和技巧非常多,具体请参考鲁鹏老师的课程\footnote{\url{https://www.bilibili.com/video/BV1Ss4y197CU}}(多视图重建一节)。\\

	这里要说的是这个\textbf{捆绑调整}方法,本质通过$L_2$距离最小化投影误差,
	$$
		E(M,X) = \sum_{i=1}^m \sum_{j=1}^n D\left(x_{ij}, M_iX_j\right)^2
	$$

	\begin{figure}[H]
		\begin{center}
			\includegraphics[width=\textwidth]{images/ba.png}
		\end{center}
		\caption{BA重投影误差}
	\end{figure}

	理论上可以通过梯度下降优化BA,直接得到$M,X$;实际上传统做法是先用增量法求的一个初解,再用BA方法优化,以加速收敛。\\

	如果样本足够多,将传统SfM转化为机器学习问题,效果也未必就差;但传统方法是对单场景重建,能收集的样本有限。\\

	图像上的点,至少被两个相机看到,构成点对的点才能计算出深度信息,如果两个相机姿态差距太大,会因为遮挡等原因导致可匹配的点减少。\\

	BA优势就是不需要所有的点都被所有的相机看到,每个相机只关注自己能看到点即可。
\section{MVSNet}

	不同于传统的SfM估计一次优化能得到整个场景的点云,MVSNet\footnote{\url{https://arxiv.org/abs/1804.02505}}只是推断每张图片的深度信息,通过后处理,把多张图片的深度,拼成场景点云。\\

	主要思路是通过2张目标图来协助推断参考图的深度值,

	\begin{enumerate}
		\item 把参考相机(第一个相机)像素平面的坐标,通过192个尺度反投影到相机坐标系;然后通过单应矩阵变换到目标相机的像素坐标系。

		\small\textit{ 注意,这里的变换对象是参考相机的$120 \times 160$像素平面坐标,而非论文说的参考图像的特征图;}

		经过这步操作,可在目标相机的像素平面上形成192个$120 \times 160$的新的像素坐标值。

		\item 把投影得到的像素坐标值与目标图像的特征图结合,得到192个目标特征图

		\small\textit{把192个特征图按深度叠在一起,形成一个$192 \times 120 \times 160 \times 32$的特征图,称为\textbf{Feature Volume}}

		\item 把多个(文中为2个)目标\textbf{Feature Volume}连同参考图像的特征图一起,计算一个方差,得到的结果称为\textbf{Cost Volume},随后在\textbf{Cost Volume}上回归参考图像的深度值

		\small\textit{注意,此时才用到参考图像的特征图}
	\end{enumerate}

	\begin{figure}[H]
		\begin{center}
			\includegraphics[width=\textwidth]{../images/fronto.png}
		\end{center}
		\caption{相机1将场景按深度切片称为\textbf{Fronto parallel},分别投影到相机2,3像素平面,形成\textbf{Feature Volume}}
	\end{figure}

\subsection{目标相机的作用}

	目标相机对预测参考图像的深度有什么帮助?我是这么认为的,因为世界坐标系选为参考相机的坐标系,如果参考图像上有一个点深度为$d$,那么其他相机看到这个点深度也应该为$d$,也就是用目标图像的特征预测出的深度也应该为$d$。\\

	因此,如果一个点的深度在各个相机下预测正确,那么方差就会很小,否则方差会变大,这是作者认为\textbf{Cost Volume}有作用的原因。\\

	把\textbf{Feature Volume}的平均作为回归目标也是可行的,但作者在消融实验提到效果不如方差,这里是个玄学。\\

	提升目标图片数量,预测效果也会提升,作者实验过5张图片效果好于2张。能提升预测效果的图片是那些与目标相机姿态差距不是很大的图片,所以这个数量也是有上限的。

\subsection{单应变换的作用}
	MVSNet工作重点便是单应变换,主要体现在:
	\begin{itemize}
		\item 通过单应变换把场景切面变换到目标相机像素坐标系
		\item 单应变换可求导,使得变换过程可训练
	\end{itemize}

	然而,变换切面通过增广矩阵的逆变换即可,矩阵变换本身可求导,顺便包含了第二点。\\

	虽然全文都在介绍单应矩阵,实现实际完全不需要这个概念,更不需要那些复杂的推导,是不是有点讽刺!\\

	需要考虑下面两个问题,对单应变换的影响,
	\begin{itemize}
		\item 不同深度的切面通过单应变换产生的像素值会有多大的差异?
		\item 单应变换产生的新坐标如何跟目标特征图结合生成新的特征图?
	\end{itemize}

	\subsubsection*{深度对像素坐标的影响}
	对第一个问题,单应矩阵$H$可以简化为,

	\begin{align*}
		H_d= K^{\prime}\left(R + T\mathbf{n}_d^T\right)K^{-1} &= \mathbf{A} +\frac{1}{d}\mathbf{B}\\
		&= \begin{bmatrix}
			\mathbf{A}_1 + \frac{1}{d}\mathbf{B}_1\\
			\mathbf{A}_2 + \frac{1}{d}\mathbf{B}_3\\
			\mathbf{A}_3 + \frac{1}{d}\mathbf{B}_4
		\end{bmatrix}
	\end{align*}

	$\mathbf{A}=K^\prime R K^{-1},\mathbf{B}=K^{\prime}T\mathbf{n}^TK^{-1}$,只有$d$是变量,
	$$
		p^\prime(d) = H_d p = \begin{bmatrix}
			\left(\mathbf{A}_1 + \frac{1}{d}\mathbf{B}_1\right)p\\
			\left(\mathbf{A}_2 + \frac{1}{d}\mathbf{B}_3\right)p\\
			\left(\mathbf{A}_3 + \frac{1}{d}\mathbf{B}_4\right)p
		\end{bmatrix}
	$$

	$p^\prime$的欧式坐标$p^\prime_e$为,
	$$
		p^\prime_e(d) = \left(
			\frac{\left(\mathbf{A}_1 + \frac{1}{d}\mathbf{B}_1\right)p}{\left(\mathbf{A}_3 + \frac{1}{d}\mathbf{B}_3\right)p},
			\frac{\left(\mathbf{A}_2 + \frac{1}{d}\mathbf{B}_3\right)p}{\left(\mathbf{A}_3 + \frac{1}{d}\mathbf{B}_3\right)p}
		\right)^T
	$$

	$$
		p^\prime_e(\infty) = \lim_{d\rightarrow \infty}p^\prime_e(d) = \left(
			\frac{\mathbf{A}_1p}{\mathbf{A}_3p},
			\frac{\mathbf{A}_2p}{\mathbf{A}_3p}
		\right)^T
	$$

	$$
		p^\prime_e(0) = \lim_{d\rightarrow 0}p^\prime_e(d) = \left(
			\frac{\mathbf{B}_1p}{\mathbf{B}_3p},
			\frac{\mathbf{B}_2p}{\mathbf{B}_3p}
		\right)^T
	$$	

	深度$d$的确影响投影的$u,v$坐标,且输出在$p^\prime_e(0)$和$p^\prime_e(\infty)$之间。\\

	但因为在两端存在极限,所以$d$在一定的区间内变化,对$u,v$才能有显著的影响,
	
	\subsubsection*{变换特征映射}
		现在考虑第二个问题,参考相机和目标相机像素平面分辨率为$120 \times 160$,超出这个区间的变换坐标怎么处理?\\

		MVSNet的是根据最大坐标值,$u^\prime = u/119, v^\prime = v/159$(坐标从0开始)后变换到$[-1,1]$,对超出此区间的坐标一律忽略。\\

		目标特征图的左上角对应$(-1,-1)$,右下角对应$(1,1)$,根据$(u^\prime, v^\prime)$的值查找特征图上对应的点(连续值),并根据周围四个坐标(离散值)做双线性插值计算具体对应值。\\

		显然,考相机和目标相机的分辨率即使不同,也不影响这个插值过程。\\

		如果超出$(120,160)$的坐标都被忽略,是否会降低学习效果?\\

		这里实际隐含一个约束,要求两个相机应该能看到尽量多的共同点,这通过图像对的选择策略来实现,称为\textbf{视角选择}。

\subsection{视角选择}
	
	\begin{figure}[H]
		\begin{minipage}[t]{0.48\linewidth}
			\centering
			\includegraphics[width=\textwidth]{../images/baseline_error.png}
			\caption{基线太小,重建误差较大}
		\end{minipage}		
		\begin{minipage}[t]{0.52\linewidth}
			\centering
			\includegraphics[width=\textwidth]{../images/track_filter.png}
			\caption{通过Track过滤小基线视角}
		\end{minipage}			
	\end{figure}

	图一小基线的场景,如果$p^\prime$误匹配$q$,$p,q$虽然只偏离了一个小角度$\theta$,重建的$P^\prime$与$P$误差也很大。\\

	反之,大基线场景会产生遮挡,极大减少可重建点数。MVSNet给出下面的评分机制来选择合适的图像对,

	$$
		\theta_{ij} = (180/\pi)\arccos\left(
			(\mathbf{c}_i - \mathbf{p})
			\cdot
			(\mathbf{c}_j - \mathbf{p})
		\right)
	$$

	$\theta_{ij}$为点对夹角,计算方法是:将像素坐标投影回世界坐标系,以第一个相机的光心为原点计算内积的反余弦(图二);\\

	\textit{相机内参和像素坐标都已知,这个过程是没问题的。}\\

	通过下面分段函数把$\theta_{ij}$拟合为打分,

	$$
		\mathcal{G}(\theta) = 
		\begin{cases}
			\exp\left(
				-\frac{(\theta-\theta_0)^2}{2\sigma_1^2}
			\right), \theta \leq \theta_0\\
			\exp\left(
				-\frac{(\theta-\theta_0)^2}{2\sigma_2^2}
			\right), \theta > \theta_0\\
		\end{cases}
	$$

	\begin{figure}[H]
		\begin{center}
			\includegraphics[width=0.9\textwidth]{../images/piece_gaussian.png}
		\end{center}
		\caption{分段高斯函数,$\theta_0=5, \sigma_1 = 1, \sigma_2 = 10$}
	\end{figure}

	该函数对小基线惩罚更大,夹角小于$5$度,走小方差分支,下降更快;大于5度,走大方差分支,下降平稳一些。\\

	最终图像对打分,
	$$
		s_{ij} = \sum_{\mathbf{p}}\mathcal{G}\left(\theta_{ij}(\mathbf{p})\right)
	$$

	匹配点对越多,且夹角不小于5度,则图像之间打分越高。

\subsection{实现细节}
	MVSNet将$1200 \times 1600$的原图下采样到$800 \time 600$,从中心crop出$512 \times 640$作为输入,经过特征提取后生成$128 \times 160 \times 32$的特征图。\\

	输出为参考图的深度值,GroundTruth也是参考图的深度值,均为$128 \times 160$。 处理分几个步骤,

	\begin{itemize}
		\item 每次输入3张图,一张为参考,2张目标,经过特征提取后,得到三个$128 \times 160 \times 32$的特征图
		
		\item 采用192个深度单位,通过单应变换将参考坐标变换到两个目标相机姿态下

			\textit{作者原始实现用了复杂的单应矩阵,非常繁琐;pytorch版本用了增广变换矩阵,非常简洁}

		\item 根据坐标变换得到两个$192 \times 128 \times 160 \times 32$的特征图,如上节描述

			\textit{按深度方向,从425mm到935mm,按2mm间隔离散成192个像平面}

		\item 连同参考特征图一起,计算三个\textbf{Feauture Volume}的方差,得到\textbf{Cost Volume};进行3D卷积,通过softmax沿着深度方向作概率归一化,结果为\textbf{Probability Volume},维度为$192 \times 128 \times 160$

		\item 沿\textbf{Probability Volume}深度方向计算深度期望值,得到维度为$128 \times 160$的深度特征图,
			$$
				D = \sum_{d = d_{min}}^{d_{max}} d \times \mathbf{P}(d)
			$$

			论文中提到,这一步实际想对\textbf{Probability Volume}作一个$\arg\max$操作,选出当前位置最可能的深度值,但是$\arg\max$无法求导,无法进行梯度回传,所以用soft $\arg\max$(原文说成是soft $\arg\min$)。

			\textit{所谓soft $\arg\max$实际就是利用了$softmax$逼近$\arg\max$的特性,以此来代替$\arg\max$\footnote{\url{https://en.wikipedia.org/wiki/Softmax_function}}}

		\item 通过$L_1$回归参考图的深度值;这里只有参考图的深度值为GroundTruth,src图的深度值未使用。
	\end{itemize}

	在增量SfM方法中,一次考虑所有图像,通过反投影距离进行优化,得到整个场景的点云;但在具体实现中,至少被三个相机看到的点,才能重建出点云;\\

	MVSNet中,训练的时候只是用了3张图片,但可以推断出参考图像上所有点的深度信息;在后处理时需要把图片两两输入,构建整个场景的点云。\\

	SfM的依赖图像之间的对应点对,具体会通过SIFT特征或深度学习的方法去构建对应关系;这一步实际也会引入很多匹配噪音,尤其对一些纹理较少的图像,能计算出的匹配点较少。\\

	MVSNet没有这一步处理,对应关系已经隐含在深度预测中,能匹配的点对,看到的深度也应该一样。\\

	事实证明,通过隐式特征构建对应关系,远远好于在RGB空间用显式特征构建对应关系。

\subsection{后处理}
	主要是对深度作两种过滤,

	\begin{itemize}
		\item \textbf{\textit{photometric约束}} \\

		对\textbf{Probability Volume}中概率小于0.8的点进行过滤;对噪音点不同的相机做出的判断方差会比较大,从而回归出的概率会变小

		\item \textbf{\textit{geometric约束}} \\
			
			点对$p,p^\prime$,以$p$图为参考图预测出深度值$d$;以$p^\prime$为参考图,预测出深度值$d^\prime$\\

			将$p^\prime$以深度$d^\prime$反投影到第一个视图,得到$p_{reproj}$,及对应的深度$d_{reproj}$

		\begin{align*}
			&\vert p - p_{reproj}| < 1\\
			&\frac{\vert d - d_{reproj}\vert}{d} < 0.01
		\end{align*}
		
		使得像素误差在1像素内,深度误差在1\%以内,文中提到这种约束至少在3视图上执行,如此同一点会有很多个重投影深度,取平均作为当前点的最终深度。
	\end{itemize}

\subsection{深度图融合}
	将不同视觉的深度图融合成场景点云,MVSNet采用“Real-time visibility-based fusion of depth maps”\footnote{\url{https://people.inf.ethz.ch/pomarc/pubs/MerrellICCV07subm.pdf}}的方法,看了下就是基于简单的遮挡规则来过滤一些不置信的点。\\

	这个文章也是作者的,虽然文中引用了自身,但code是在他人基础上改的,思路是下面第一篇。。。

	\begin{itemize}
		\item  Massively parallel multiview stereopsis by surface normal diffusion\footnote{\url{https://www.cv-foundation.org/openaccess/content_iccv_2015/papers/Galliani_Massively_Parallel_Multiview_ICCV_2015_paper.pdf}}
		\item Pixelwise view selection for unstructured multi-view stereo(COLMAP)\footnote{\url{https://demuc.de/papers/schoenberger2016mvs.pdf}}
		\item Structure-from-Motion Revisited\footnote{\url{https://openaccess.thecvf.com/content_cvpr_2016/papers/Schonberger_Structure-From-Motion_Revisited_CVPR_2016_paper.pdf}}
	\end{itemize}

	后两篇文章是COLMap的基础,MVSNet也有间接参考。

\subsection{问题\&改进}
	为什么要将场景切成192片,作单应变换去构建Cost,而不是直接通过基础矩阵变换构建Cost?\\

	单应变换是传统处理常用的手段,这个文章发表比较早,受传统MVS影响比较大,可能当时还没有建立端到端的思路,或者是端到端没做出效果。\\

	切片思路得假设目标场景最大最小距离,切片多了影响计算性能,并且有很多无意义的切片;切片少了又影响精度,对此"Cascade Cost Volume for High-Resolution Multi-View Stereo and Stereo Matching"\footnote{\url{https://arxiv.org/abs/1912.06378}}这个文章专门提了一些切片的优化方法,后面我们再细说。

\section{训练集结构}
	
	\subsection*{DTU数据集} 
		mvsnet用的是DTU数据集,DTU数据集通过移动的机械臂在$49$或$64$个位置,用$7$种光照,通过结构光扫描了$124$个场景,生成$1200 \times 1600$像素的图片及对应的点云信息。\\

		因此,每个场景有$49 * 7 = 343$张照片,整体数据集包含$42532$张照片及对应的点云信息。因为机械臂移动受到严格控制,所以每张图片都有对应的高精度相机参数。\\

		mvsnet训练集用了$79$个场景,共计$27097$张图片;测试集用了$22$个场景,共计$7546$张图片。
	
	\subsection*{深度信息}
		MVSNet的输入的是每个点的深度,通过泊松表面重建,将DTU数据集的点云转变为mesh,计算出每个像素点对应的深度信息。

	\subsection*{数据集结构}
		\begin{itemize}
			\item mvs{\_}training$\slash$dtu$\slash$Cameras$\slash$pair.txt 记录了$49$个位置,每个位置的照片与其他$10$张照片的匹配程度;注意这个pair内容与场景无关,仅与相机位置有关。因为相机位置受到精准控制,所以通过空间关系可以描述出位置之间的差异。

			\item mvs{\_}training$\slash$dtu$\slash$Cameras{\_}train$\slash$000000\{YY\}{\_}cam.txt 存放相机内外参数及尺度缩放信息

			\item mvs{\_}training$\slash$dtu$\slash$Rectified$\slash$scan\{XX\}{\_}train 训练样本,XX为场景编号
				\begin{itemize}
					\item rect{\_}0\{YY\}{\_}\{Z\}{\_}r5000 为场景scanXX对应的图片,YY为相机位置编号$ 1 \sim 49$,Z为光照强度编号$0 \sim 6$
				\end{itemize}

			\item mvs{\_}training$\slash$dtu$\slash$Depths$\slash$scan\{XX\}{\_}train$\slash$
				\begin{itemize}
					\item depth{\_} map{\_}00\{YY\}.pfm 场景scanXX对应相机位置YY($0 \sim 48$)的深度图,格式为pfm
					\item depth{\_} map{\_}00\{YY\}.png 深度图的可视化
				\end{itemize}
			
		\end{itemize}	

\bibliography{math}
\end{document}