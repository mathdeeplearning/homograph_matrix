\section{Shannon采样定理}

\subsection{采样定理}

如果已知某个连续函数$f(x)$一些离散值$f(x_i)(i=1,2,\dots)$,能否恢复出完整的$f(x)$?\\

如果$f(x)$本身及对应的傅里叶变换符合一定的条件,这是可以做到的,有如下\textit{Shannon采样定理},

\begin{theorem}\label{th:discret_rep}
	\textbf{Shannon采样定理}\quad 若$f \in L_{1}(\mathbb{R})$,$\hat{f}$的支撑集为$[-B,B]$($B$称为\textit{Nyquist 频率}),则
	\begin{equation}\label{shannon_th}
		f(x) = \sum_{n \in \mathbb{Z}} f\left(\frac{n}{2B}\right) \mathop{sinc}\left(2B\left(x-\frac{n}{2B}\right)\right)
	\end{equation}
\end{theorem}

分两步证明,

\begin{enumerate}
	\item 将$\hat{f}$作傅里叶级数展开(需要$\hat{f}$紧支撑)
	\item 作$\hat{f}$的逆变换得到$f$
\end{enumerate}

\begin{proof}
	$\hat{f}$支撑集是$[-B,B]$,周期延拓后作傅里叶展开,分量系数为(\ref{f_series_form1}),
	\[
		c_n = \frac{1}{2B} \int_{-B}^B \hat{f}(x) e^{\frac{i2\pi  n x}{2B}}dx = \frac{1}{2B}f\left(\frac{-n}{2B}\right)
	\]
	
	对应的傅里叶级数,
	\begin{equation*}
		\hat{f}(\omega) = \sum_{n \in \mathbb{Z}}\frac{1}{2B}f\left(\frac{n}{2B}\right)e^{\frac{-i2\pi  n \omega}{2B}}
	\end{equation*}

	通过窗函数把定义域扩展到$(-\infty,\infty)$,
	
	\begin{equation}\label{eq:freq_series}
		\hat{f}(\omega) = \sum_{n \in \mathbb{Z}}\frac{1}{2B}f\left(\frac{n}{2B}\right)
		\chi_{[-B,B]}(\omega)
		e^{\frac{-i2\pi  n \omega}{2B}}
	\end{equation}

	结合傅里叶变换的性质:
	\begin{itemize}
		\item 周期性:$f(-x) = \mathcal{F}(\hat{f})(x)$
		\item 窗函数对应sinc函数:$\hat{\chi}_{[-B,B]}= 2B\mathop{sinc}(2Bx)$
		\item 旋转对应平移:乘以$e^{\frac{-i2\pi  n \omega}{2B}}$等价平移$-\frac{n}{2B}$
	\end{itemize}

	对(\ref{eq:freq_series})两边做傅里叶变换可得(\ref{shannon_th})。
\end{proof}

采样定理表明,$f(x)$可以用$\mathop{sinc}$函数基分解,系数为离散采样点,来自傅里叶级数展开。

\subsection{Nyquist采样频率}
	
	用任意的Nyquist频率$B_0$替换(\ref{eq:freq_series})中的$B$,得到频谱$\hat{h}$,
	$$
		\hat{h}(\omega) = \sum_{n \in \mathbb{Z}}\frac{1}{2B_0}f\left(\frac{n}{2B_0}\right)
		\chi_{[-B_0,B_0]}(\omega)
		e^{\frac{-i2\pi  n \omega}{2B_0}}
	$$

	现在的问题是,$\hat{h}$对应的信号$h$与原始信号$f$有什么关系?\\

	下面按$B_0$与$B$的关系讨论一下,\\

	\textbf{1、$B_0 < B$}, 对任意$|\omega| \in [B_0,B]$,都有$\hat{h}(\omega) =0 \ne \hat{f}(\omega)$,此时$\hat{h}$一定无法重建出$f(x)$\\

	\textbf{2、$B_0 > B$},扩展$\hat{f}$的支撑集到$[-B_0,B_0]$,构建函数$\tilde{h}$,
	$$
		\tilde{h}(\omega) = \chi_{[-B,B]}\hat{f}(\omega), \quad\omega \in [-B_0, B_0] 
	$$

	将$\tilde{h}$以$2B_0$作周期延拓,展开傅里叶级数的系数为,
	\begin{align*}
		c_n
			&= \frac{1}{2B_0}\int_{-B_0}^{B_0} \chi_{[-B_0,B_0]}\hat{f}(x) e^{\frac{i2\pi nx}{2B_0}}dx\\
			&= \frac{1}{2B_0}\int_{-B}^{B} \chi_{[-B,B]}\hat{f}(x) e^{\frac{i2\pi nx}{2B_0}}dx\\
			&= \frac{1}{2B_0}\int_{-B}^{B} \hat{f}(x) e^{\frac{i2\pi nx}{2B_0}}dx\\
			&= \frac{1}{2B_0}f\left(\frac{n}{2B_0}\right)
	\end{align*}

	对比可知,$\hat{h} = \tilde{h}$,所以$\hat{h}$是$\hat{f}$支撑集扩大到$[-B_0,B_0]$后的傅里叶级数,这表明函数傅里叶级数的分解不是唯一的。\\

	$\hat{h}$的逆变换,根据周期性,

	\begin{align*}
		h(-x) &= \mathcal{F}(\hat{h})(x) 
			= \left(\int_{-B_0}^{-B} + \int_{-B}^{B} + \int_{B}^{B_0} \right)
			\chi_{[-B,B]}\hat{f}(\omega) e^{-i2\pi \omega x}d\omega
			 = f(-x)
	\end{align*}
	
	\textit{当$B_0\ge B$,通过Shannon采样定理构造的函数,能点点收敛到原函数。也就是采样频率区间至少是$2B$。}\\

	证明过程蕴含了一个反直觉的结论:

	\begin{itemize}
		\item 时域空间,高采样频率对应更密集的样本,对信号重建\underline{\textit{应该有}}帮助
		\item 频域空间,高采样频率只是扩展了频谱支撑集,对频谱重建\underline{\textit{没有}}帮助
	\end{itemize}
	
	时域与频域是同构,在频域没有帮助的操作,对应到时域就是冗余计算。在工程实践中,信号存在噪音、丢失等情况,更密集的采样有助于平滑这些异常值。\\

	一般无法知道信号的最高频率,经常根据经验估一个采样频率,估低了高频信息会丢失,无法准确重建;估高了又会造成冗余计算,这是一个权衡的过程。

\subsection{sinc正交基}
	
	Shannon采样定理(\ref{th:discret_rep})揭示了一个深刻的结论:对频域有\textit{紧支撑}的连续函数,可以sinc函数基分解,分量是原函数的离散值。\\

	这里要说明的是,sinc函数基是标准正交基,是非常理想的函数坐标系。\\

	注意一个通过傅里叶变换计算积分的技巧,
	$$
		\int_{-\infty}^{\infty} f(x) dx = \hat{f}(0)
	$$

	令
	$$
		J(n,\omega) = \mathcal{F}(\mathop{sinc}(x)\mathop{sinc}(x+n)) (\omega)\quad (n\in \mathbb{Z})
	$$
	只要证明$J(n,0)=0,J(0,0) = 1$即可。

	\begin{align*}	
		J(n,\omega) &= \mathcal{F}(\mathop{sinc}(x)) \otimes \mathcal{F}((\mathop{sinc}(x+n)) \\
		& = \chi(\omega) \otimes e^{-i2\pi \omega n} \chi(\omega) \\
		& = \int_{-1/2}^{1/2} e^{-i2\pi (\omega -x) n} dx\\
	\end{align*}

	$$
		\begin{aligned}
				J(n,0)&= \int_{-1/2}^{1/2} e^{i2\pi nx } dx =0 \\
				J(0,0)&= \int_{-1/2}^{1/2} dx = 1		
		\end{aligned}
	$$

	进一步,函数可以展开为某个\textit{核函数}$\psi$的级数($T$为采样周期),
	$$
		f(x) = \sum_{n\in \mathbb{Z}}f(nT)\psi(ax-nT) 
	$$	

	除了sinc函数,核函数还有很多的选择,比如高斯函数、各类小波基函数。不同的基函数,能观察到函数(信号)不同的性质,
	
	\begin{itemize}
		\item 傅里叶基,能观察到信号频率的构成情况;
		\item sinc函数基,能观察到函数通过离散点重建的过程;
		\item 小波基,能观察到随观察尺度变化的函数响应
	\end{itemize}

	\textit{向量空间}不同的向量基之间存在一个\textit{线性变换},常用矩阵表示;\textit{函数空间}的函数基之间虽然也存在某种\textit{非线性}变换,但却没有统一的表示形式。\\

	另外,机器学习的核心工作是数据降维,实际就是以数据驱动的方式学习出合适的坐标系,但坐标系是以网络参数的形式存在,无法形式化分析,这种不可解释的黑盒坐标系,在实际任务中却表现的很好。
	




	





