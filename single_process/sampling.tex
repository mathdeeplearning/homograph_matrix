\section{信号采样}

信号采样主要目的是把连续信号给存入计算机进行分析,目前的计算机体系无论是存储还是计算,都无法处理连续数据,所以需要对连续数据做离散化。\\

离散化的手段是每隔一段时间(周期)捕捉一下连续信号的瞬时值,记录并存储。那自然就面对两个问题:如何捕捉瞬时值、如何根据离散值恢复原始信号。\\

理想的捕捉方式是周期性发射冲激信号,通过冲激信号刺激原信号来达到获取原信号瞬时值的目的,实操上会制作一些工具,比如收音机这种;我们只讨论数学上的实现,这个冲击信号就是$\delta$函数(\ref{delta_func}),$\delta$函数的\textit{选择性}保证能准确取到原信号的瞬时值。\\

冲激信号周期性发射,就构成Dirac梳函数(\ref{dirac_comb}),信号的采样转变为梳函数与信号的积分。

\subsubsection*{频谱混叠}

梳函数与信号的积分,在频域中表示复制信号频谱并且求和(图\ref{comb_replica}),当采样频率过低(低于信号频率区间$2B$),会导致\textit{频谱混叠},在混叠区域可能导致信号无法准确重建。\\

但频谱混叠并无法定量描述带来的问题,借助\textit{Shannon采样定理}(\ref{th:discret_rep}),更清楚认清采样的本质,
\begin{itemize}
	\item 采样本质是低通滤波,频率越低,漏掉的高频信息越多,所以低于信号自身最高频率的采样,一定无法重建出原始信号
	\item 更高的采样频率只是扩展了信号频率的支撑集合,对频率重建及信号重建没有任何帮助
\end{itemize}

权衡信号重建误差及计算开销,只有按信号固有频率采样才是最优操作。

\subsubsection*{采样定理}

\textit{Shannon采样定理}是把函数在sinc基下展开,分量为信号离散采样值,这个定理在\textit{离散}和\textit{连续}之间架起了一座神奇的桥梁。\\

在采样定理证明过程中发现,采样定理把傅里叶变换和傅里叶级数巧妙的联系在了一起。\\

当然了,站在实分析的角度,形如
$$
	f(x) = \sum_{n\in \mathbb{Z}}f(nT)\phi(x-nT)
$$

的分解有非常多。