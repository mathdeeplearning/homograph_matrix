\section{傅里叶级数、傅里叶变换}

\subsection{傅里叶级数}

周期为$2\pi$的连续函数$f(x)$,可展开为傅里叶级数(\textit{几乎处处}),

$$
	f(x) = \frac{a_0}{2} +\sum_{n=1}^{\infty}
	\left(
		a_n \cos nx + b_n\sin nx
	\right)
$$

其中,
\begin{align*}
	a_0 &= \frac{1}{\pi}\int_{-\pi}^{\pi} f(x)dx\\
	a_n &= \frac{1}{\pi}\int_{-\pi}^{\pi} f(x)\cos nx dx\\
	b_n &= \frac{1}{\pi}\int_{-\pi}^{\pi} f(x)\sin nx dx\\
\end{align*}

这是傅里叶级数的基本形式,可通过复指数进一步简化,考虑欧拉公式,
$$
	e^{i\theta} = \cos \theta + i \sin \theta
$$

可得,

\begin{align*}
	\cos\theta = \frac{e^{i\theta} + e^{-i\theta}}{2}\\
	\sin\theta = \frac{e^{i\theta} - e^{-i\theta}}{2i}\\
\end{align*}

所以,
\begin{align*}
	a_n \cos nx + b_n\sin nx 
		&= a_n\cdot\frac{e^{inx} + e^{-inx}}{2} + b_n\cdot\frac{e^{inx} - e^{-inx}}{2i}\\
		&= \frac{a_n - ib_n}{2}\cdot e^{inx} + \frac{a_n + ib_n}{2}\cdot e^{-inx}
\end{align*}

傅里叶级数可表示为,

\begin{align*}
	f(x) &= \frac{a_0}{2} + \sum_{n=1}^{\infty}
		\left(
			\frac{a_n - ib_n}{2}\cdot e^{inx} + \frac{a_n + ib_n}{2}\cdot e^{-inx}
		\right)\\
		&= \frac{a_0}{2} + \sum_{n=1}^{\infty}\frac{a_n - ib_n}{2}\cdot e^{inx} 
		+ \sum_{n=1}^{\infty}\frac{a_n + ib_n}{2}\cdot e^{-inx}\\
		&= \frac{a_0}{2} + \sum_{n=-\infty}^{-1}\frac{a_{-n} - ib_{-n}}{2}\cdot e^{-inx} 
		+ \sum_{n=1}^{\infty}\frac{a_n + ib_n}{2}\cdot e^{-inx}\\
		&= \sum_{n=-\infty}^{\infty}c_ne^{-inx}
\end{align*}

将$n$从正整数扩展到整数集合,得到了简洁的复指数表示。但还是不够简洁,因为系数$c_n$需分段表示,

\begin{align*}
	c_n &= \frac{a_n + ib_n}{2}, \quad n=1,2,3,\dots\\
	c_n &= \frac{a_0}{2}, \quad n=0\\
	c_n &= \frac{a_n - ib_n}{2}, \quad n=-1,-2,-3,\dots
\end{align*}

考虑$n$为正整数的场景,
\begin{align*}
	c_n &= \frac{a_n + ib_n}{2}\\
		&= \frac{1}{2\pi}
			\left[
				\int_{-\pi}^{\pi}f(x)\cos nx dx + i\int_{-\pi}^{\pi}f(x)\sin nx dx
			\right] \\
		&= \frac{1}{2\pi}\left[
			\int_{-\pi}^{\pi}f(x)
			\left(
				\frac{e^{inx}+e^{-inx}}{2}
			\right) dx 
			+ i\int_{-\pi}^{\pi}f(x)
			\left(
				\frac{e^{inx} -e^{-inx}}{2i}
			\right) dx
		\right]\\
		&=\frac{1}{2\pi}\int_{-\pi}^{\pi}f(x)e^{inx}dx
\end{align*}

当$n$为负整数时,显然$c_{-n} = c_n$,级数进一步简化为,

\begin{align}
	f(x) &= \sum_{n=-\infty}^{\infty}c_n e^{-inx}\\
	c_n &= \frac{1}{2\pi}\int_{-\pi}^{\pi}f(x)e^{inx}dx
\end{align}


若$f$的周期为$2L$,对应积分区间$[-L,L]$,令$z = \frac{\pi}{L}x$的周期为$2\pi$,对应傅里叶级数为,

$$
	f(x) = f(\frac{L}{\pi}z) = g(z)
$$

可验证,$g(z)$的周期为$2\pi$,作傅里叶展开可得,
\begin{equation}\label{f_series_form1}
	\begin{aligned}
		f(x)= &\sum_{n=-\infty}^{\infty}c_ne^{-in\frac{\pi}{L}x}\\
		c_n =& \frac{1}{2L}\int_{-L}^L f(x) e^{in\frac{\pi}{L}x}dx
	\end{aligned}
\end{equation}
因为$c_n = c_{-n}$,也可表示为,

\begin{equation}
	\begin{aligned}\label{f_series_form2}
		f(x)= &\sum_{n=-\infty}^{\infty}c_ne^{in\frac{\pi}{L}x}\\
		c_n =& \frac{1}{2L}\int_{-L}^L f(x) e^{-in\frac{\pi}{L}x}dx
	\end{aligned}
\end{equation}

傅里叶变换常解释为把一个连续信号分解为不同频率的正余弦信号叠加,分量$c_ne^{-in\pi x/L}$,表示频率为$2L/n$的信号的幅度为$c_n$。

\subsection{傅里叶变换}

	傅里叶级数能把周期信号分解为各种频率的正余弦信号,但非周期信号则无法分解,但我们又想了解非周期信号的频率成分,如此便有\textit{傅里叶变换}。\\

	非周期信号,可以认为周期无限大,即$L\rightarrow \infty$,

	(\ref{f_series_form2})式可重写为,
	$$
		2L c_n = \int_{-L}^L f(x) e^{-in\frac{\pi}{L}x}dx
	$$

	$n$可取任意值,当$L\rightarrow \infty$时,$n/2L$是一个连续变量,记为$\omega$,实际就是频率分量;则上式重写为
	$$
		2L c_n = \int_{-L}^L f(x) e^{-i2\pi\omega x}dx
	$$	

	两边取极限,

	$$
		\lim_{L\rightarrow\infty} 2L c_n = \lim_{L\rightarrow\infty} \int_{-L}^L f(x) e^{-i2\pi\omega x}dx
	$$	

	只要$f$符合\textit{绝对可积},右边极限是存在的,则左边极限也存在,且是$\omega$的函数,记为$\hat{f}(\omega)$,
	
	\begin{equation}
		\hat{f}(\omega) =  \int_{-\infty}^{\infty} f(x) e^{-i2\pi\omega x}dx\label{f_trans}
	\end{equation}

	$\hat{f}$为函数$f$的\textit{傅里叶变换},$\hat{f}$的变量是频率$\omega$,值是该频率对应的幅度。\\

	\textit{\textbf{绝对可积}}条件,
	$$
		\int_{-\infty}^{\infty} |f(x)|dx < \infty
	$$

	同样,如果已知$\hat{F}$,可以通过逆变换得到原始信号$f$,

	\begin{equation}
		f(x) =  \int_{-\infty}^{\infty} f(\omega) e^{i2\pi\omega x}d\omega\label{inver_f_trans}
	\end{equation}

	常用$\mathcal{F},\mathcal{F}^{-1}$表示傅里叶变换及逆变换,则上述变换可简记为,
	\begin{align*}
		\hat{f}(\omega) &= \mathcal{F}(f) = \int_{-\infty}^{\infty} f(x) e^{-i2\pi\omega x}dx\\
		f(x) &= \mathcal{F}^{-1}(\hat{f}) = \int_{-\infty}^{\infty} f(\omega) e^{i2\pi\omega x}d\omega
	\end{align*}

	\textit{在不同的文献中,逆变换前面可能出现$\frac{1}{2\pi}$系数,这个系数是否存在于指数中是否包含$2\pi$有关,如果指数中包含,则系数中没有,反之亦然。}

\subsection{傅里叶变换性质}	
	\subsubsection*{1、线性} 
		$$
			\mathcal{F}(\alpha f + \beta g) = \alpha \mathcal{F}(f) + \beta \mathcal{F}(g)
		$$
	\subsubsection*{2、时域平移相当于频域旋转}
		$$
			\mathcal{F}(f(x+a)) = e^{-j2\pi\omega a}\mathcal{F}(f)
		$$
	\subsubsection*{3、尺度性质}
		$$
			\mathcal{F}(f(ax)) = \frac{1}{|a|}\hat{f}(\frac{\omega}{a})
		$$

		据此性质可知,当$a>0$时,
		$$
			\mathcal{F}(\mathop{sinc}(2ax)) = \frac{1}{2a}\mathcal{F}(\mathop{sinc}(\frac{x}{2a}))
		$$

		是一个定义在$[-a,a]$,值为$\frac{\pi}{2a}$的窗函数。

	\subsubsection*{4、时域卷积相当于频域乘积,时域乘积相当于频域卷积}
		\begin{align*}
			\mathcal{F}(f \otimes g) &= \mathcal{F}(f) \mathcal{F}(g)\\
			\mathcal{F}(fg) &= \mathcal{F}(f) \otimes \mathcal{F}(g)
		\end{align*}

		此性质带来极大的便利,比如时域卷积不好计算,可以计算频域乘积,然后做逆傅里叶变换:
		$$
			f*g =\mathcal{F}^{-1}\left(\mathcal{F}(f) \mathcal{F}(g)\right)
		$$
	\subsubsection*{5、Parseval等式,又称能量守恒}
		\begin{equation}\label{parseval_eq}
			\begin{aligned}
				\int |f(x)|^2dx &= \int |\hat{f}(w)|^2d\omega\\
				\int f(x)g(x)dx &= \int \mathcal{F}(f)\mathcal{F}(g)d\omega\\
				\int \left|f(x) - g(x)\right|^2dx &= \int \left|\hat{f}(x) - \hat{g}(x)\right|^2d\omega
			\end{aligned}
		\end{equation}
		
		最后一个式子表明,时域频域的$L_2$距离保持不变。
	\subsubsection*{6、$n$阶导数的傅里叶变换}
		$$
			\mathcal{F}(f^{(n)}) = (i2\pi\omega)^n \hat{f}(\omega)
		$$
	\subsubsection*{7、周期性}
		$$
			\mathcal{F}^0 = I, 
			\mathcal{F}^1 = \hat{f}, 
			\mathcal{F}^2 = f(-x), 
			\mathcal{F}^3 = \mathcal{F}^{-1},
			\mathcal{F}^4 = \mathcal{F}^0
		$$

\subsection{特殊函数及其变换}
	\subsubsection*{1、高斯函数}
		$$ f(x) = e^{-ax^2} $$

		\begin{align*}
			\hat{f}(\omega) 
				&= \int_{-\infty}^{\infty} e^{-ax^2}\cdot e^{-i2\pi\omega x}dx\\
				&= \int_{-\infty}^{\infty}  
					\exp\left\lbrace
						-a\left(
							x^2 + 2\frac{i\pi\omega}{a} x
						\right)
					\right\rbrace dx\\
				&=  \int_{-\infty}^{\infty}  
					\exp\left\lbrace
						-a\left(
							x^2 + 2\frac{i\pi\omega}{a} x + \frac{i\pi\omega}{a}
						\right)^2 + a\left(\frac{i\pi\omega}{a}\right)^2
					\right\rbrace dx\\
				&= C\int_{-\infty}^{\infty} \exp \left\lbrace
						-a\left(
							x+\frac{i\pi\omega}{a}
						\right)^2
					\right\rbrace dx\quad ,
					\left(
						C = \exp\left\lbrace
								-\frac{(\pi\omega)^2}{a}
					\right\rbrace\right)\\
				&= C\int_{-\infty}^{\infty} \exp \left\lbrace
						-\left(
							\sqrt{a}x+\frac{i\pi\omega}{\sqrt{a}}
						\right)^2
					\right\rbrace dx\\
				& = C\int_{-\infty}^{\infty} 
							e^{-u^2}
					 \frac{1}{\sqrt{a}}dx\quad , \left(u = \sqrt{a}x+\frac{i\pi\omega}{\sqrt{a}}\right)\\
				&= \frac{\sqrt{2\pi}C}{\sqrt{a}}
					\frac{1}{\sqrt{2\pi}}\int_{-\infty}^{\infty} e^{-u^2}dx\\
				&= \sqrt{\frac{2\pi}{a}}
							\exp
								\left\lbrace
									-\frac{(\pi\omega)^2}{a}
								\right\rbrace
		\end{align*}

		高斯函数傅里叶变换还是高斯函数,无论在时域还是频域,高斯函数形式都不变。
	\subsubsection*{2、窗函数}

		标准窗函数定义为,
		$$
			\chi(x) = 
			\begin{cases}
				1,& x \in [-\frac{1}{2},\frac{1}{2}]\\
				0,& \text{others}
			\end{cases}
		$$

		以及带参数的形式,
		\begin{equation}\label{rect_pram_trans}
			\chi_{[-B,B]}(x) = \chi(\frac{x}{2B})
		\end{equation}

		\begin{align*}
			\hat{\chi}(\omega) 
				& = \int_{-\infty}^{\infty} f(x) e^{-i2\pi\omega x}dx\\
				& = \int_{-\frac{1}{2}}^{\frac{1}{2}} e^{-i2\pi\omega x}dx\\
				& = -\frac{1}{i2\pi\omega}
					\int_{-\frac{1}{2}}^{\frac{1}{2}} e^{-i2\pi\omega x}d(-i2\pi\omega x)\\
				& = \frac{1}{i2\pi\omega}\int_{-i\pi\omega}^{i\pi\omega} e^{u}d(u)\\
				&= \frac{1}{i2\pi\omega}\left(e^{i\pi\omega} - e^{-i\pi\omega}\right)\\
				&=\mathop{sinc}(\omega)
		\end{align*}

		窗函数的傅里叶变换是$\mathop{sinc}$函数。由缩放性质,

		\begin{align*}
			\hat{\chi}_{[-B,B]}(\omega) &= 2B\mathop{sinc}(2B\omega)
		\end{align*}

		根据周期性,
		$$
			\chi(-x) =\mathcal{F}^2(\chi) = \mathcal{F}(\mathop{sinc}) 
		$$
		
		sinc函数的傅里叶变换为窗函数。

	\subsubsection*{3、sinc函数}

		一般$\mathop{sinc}$函数定义为,
		$$
			f(x) =\frac{\sin x}{x}
		$$

		也经常用下面的形式,

		$$
			\mathop{sinc}(x) =f(\pi x) = \frac{\sin \pi x}{\pi x}
		$$

		$x = 0$是sinc函数的奇点,

		$$
			\lim_{x\rightarrow 0} \frac{\sin x}{x} = 1
		$$

		补充$f(x) = 1$,使得函数没有奇点。\\

		\begin{figure}[H]
			\begin{center}
				\includegraphics[width=0.8\textwidth]{./images/sinc_graph.png}
			\end{center}
		\end{figure}

		通过周期性已知sinc函数的傅里叶变换是窗函数,下面介绍下直接求解过程。\\


		考察下以下定积分(sinc是偶函数),

		$$
			J = \int_{-\infty}^{\infty} \frac{\sin x}{x}dx = 2\int_{0}^{\infty} \frac{\sin x}{x}dx
		$$

		构造$x$的积分表示,
		$$
			\frac{1}{x} = \int_0^\infty e^{-ax}da
		$$

		代入计算,

		$$
			J = 2\int_{0}^{\infty} \sin x \int_0^\infty e^{-ax}da dx = 2\int_{0}^{\infty}\int_{0}^{\infty}\sin x e^{-ax} dx da
		$$

		\begin{align*}
			K &= \int_{0}^{\infty}\sin x e^{-ax} dx\\
				&= -\int_{0}^{\infty}e^{-ax} d\cos x\\
				&= -\cos x \cdot e^{-ax}\Big|_0^{\infty} + \int_{0}^{\infty}\cos x de^{-ax}\\
				&= 1 - a \int_0^\infty \cos x e^{-ax} dx\\
				&= 1 - a^2K
		\end{align*}

		所以
		$$
			K = \frac{1}{1+ a^2}
		$$

		$$
			J = 2\int_{0}^{\infty} \frac{1}{1+a^2}da = 2\arctan(a)\Big|_0^\infty = \pi
		$$

		由此可知,
		\begin{equation}
			\int_{-\infty}^{\infty} \mathop{sinc}(x)dx = \int_{-\infty}^{\infty} \frac{\sin \pi x}{\pi x}dx = \frac{J}{\pi} = 1 \label{sinc_int_value1}
		\end{equation}

		sinc函数的傅里叶变换,
		\begin{align*}
			\mathcal{F}(\mathop{sinc}) 
				& = \int_{-\infty}^{\infty} \frac{\sin \pi x}{\pi x} e^{-i2\pi\omega x}dx\\
				&= \int_{-\infty}^{\infty} \frac{\sin \pi x}{\pi x} \cos 2\pi\omega x dx + i \int_{-\infty}^{\infty} \frac{\sin \pi x}{\pi x} \sin 2\pi\omega x dx\\
				&= 2 \int_0^\infty\frac{\sin \pi x  \cdot \cos 2\pi\omega x}{\pi x} dx\\
				&= \int_0^\infty\frac{\sin (1 + 2\omega)\pi x  + \sin (1 - 2\omega)\pi x}{\pi x} dx\\
				&= \int_0^\infty\frac{\sin (1 + 2\omega)\pi x }{\pi x} dx
				+ \int_0^\infty\frac{\sin (1 - 2\omega)\pi x}{\pi x} dx
		\end{align*}

		由(\ref{sinc_int_value1})知,
		\begin{equation}
			\mathcal{F}(\mathop{sinc})(\omega) = \chi(\omega)\label{sinc_f_trans}
		\end{equation}

		sinc函数的傅里叶变换是窗函数。

	\subsubsection*{4、Dirac $\delta$函数}
	
		Dirac $\delta$函数是一个广义函数,满足如下条件,
		\begin{enumerate}
			\item 基础性质
				\begin{equation}\label{delta_func}
				\begin{aligned}
					\delta(x) = 
					\begin{cases}
						\infty, &x=0\\
						0, &x\neq 0
					\end{cases}
				\end{aligned}				
				\end{equation}

			\item 积分性质
				$$
					\int_{-\infty}^{\infty}\delta(x)dx = 1
				$$			
		\end{enumerate}

		$\delta$函数的积分具有\textit{选择性},

		\begin{align*}
			\int_{-\infty}^{\infty}\delta(x-x_0)f(x)dx = f(x_0)\label{delta_selection}
		\end{align*}

		根据选择性,$\delta(x-x_0)$的傅里叶变换为,
		\begin{align*}
			\mathcal{F}(\delta(x-x_0))(\omega) 
				= \int_{-\infty}^{\infty} \delta(x-x_0) e^{-i2\pi\omega x}dx
				= e^{-i2\pi\omega x_0}
		\end{align*}

		$\delta(x)$的傅里叶变换为,$\mathcal{F}(\delta(x)) = 1$。\\


		$\delta(x)$的逆变换为,

		\begin{equation}
			\delta^\prime(x) = \mathcal{F}^{-1}(\mathcal{F}(\delta(x))) = \int_{-\infty}^{\infty}e^{i2\pi x \omega}d{\omega}\label{delta_int_form}
		\end{equation}

		这是$\delta$函数的积分形式,需验证$\delta^\prime$满足$\delta$的条件,才能确保逆变换的正确性。\\

		1、验证$0$点的值
		$$
			\delta(0) = \int_{-\infty}^{\infty} 1 d\omega = \infty
		$$

		2、验证非$0$点值
			\begin{align*}
				\delta^\prime(x) 
					&= \lim_{n\rightarrow \infty} = \int_{-n\pi}^{n\pi}e^{i2\pi x \omega}d{\omega}\\
					&= 2\lim_{n\rightarrow \infty}\int_{-n\pi}^{n\pi} \cos (2\pi x \omega) d{\omega}\\
					&=0
			\end{align*}
		
		3、验证积分
		\begin{align*}
			\int_{-\infty}^{\infty} \delta^{\prime}(x) dx 
				&= \int_{-\infty}^{\infty}\int_{-\infty}^{\infty}e^{i2\pi x \omega}d{\omega}dx\\
				&= 2\int_{-\infty}^{\infty}\left[\int_{0}^{\infty} \cos 2\pi x\omega d{\omega}\right]dx\\
				&= 2\int_{-\infty}^{\infty}\left[\lim_{N\rightarrow \infty}\int_{0}^{N} \cos 2\pi x\omega d{\omega}\right]dx\\
				&= 2\lim_{N\rightarrow \infty}\int_{-\infty}^{\infty}\frac{\sin 2\pi x N}{2\pi x}dx\\
				&= 1
		\end{align*}

		所以$\delta^\prime$的确是$\delta$函数。

	\subsubsection*{5、Dirac梳函数}

		Dirac梳函数是通过$\delta$函数平移$T$复合而成,

		\begin{equation}
			\mathop{C}(x,T) = \sum_{n=-\infty}^{\infty} \delta(x-nT) \label{dirac_comb}
		\end{equation}

		函数支撑集仅在$nT$处,所以是一$T$为周期的周期函数,

		\begin{figure}[H]
			\begin{center}
				\includegraphics[width=0.8\textwidth]{./images/dirac_comb.png}
			\end{center}
			\caption{周期为$T$的$\delta$梳函数}
		\end{figure}

		梳函数可以展开为\textit{傅里叶级数},根据(\ref{f_series_form2})系数为,
		$$
			c_n = \frac{1}{T}\int_{kT-T/2}^{kT+T/2}
				\sum_{k=-\infty}^{\infty} \delta(x-kT)e^{\frac{-i2\pi n}{T}x}dx = \frac{1}{T}
		$$

		\textbf{梳函数的傅里叶级数},
		\begin{equation}
			\mathop{C}(x,T) =\frac{1}{T} \sum_{n=-\infty}^{\infty} e^{\frac{i2\pi n}{T}x} \label{delta_f_series}
		\end{equation}

		\textbf{梳函数的傅里叶变换},
		\begin{align*}
			\mathcal{F}(\mathop{C(x,T)}) 
				&= \int_{-\infty}^{\infty} \frac{1}{T} \sum_{n=-\infty}^{\infty} e^{\frac{i2\pi n}{T}x} \cdot  e^{-i2\pi\omega x}dx\\
				& = \frac{1}{T} \sum_{n=-\infty}^{\infty} \int_{-\infty}^{\infty} e^{i2\pi(\omega - \frac{n}{T}) x}dx\\
				& = \frac{1}{T} \sum_{n=-\infty}^{\infty} \delta(\omega,\frac{n}{T})\\
				& = \frac{1}{T}\mathop{C}(\omega, \frac{n}{T})
		\end{align*}

		倒数第二步,
		$$
			h(\omega) = \int_{-\infty}^{\infty} e^{i2\pi(\omega - \frac{n}{T}) x}dx
		$$

		根据分步骤积分容易计算,
		$$
			h(\omega) =
			\begin{cases}
				0, & \omega = \frac{n}{T}\\
				\infty, & \text{others}
			\end{cases}
		$$

		这实际就是$\delta(\omega,n/T)$,所以梳函数的傅里叶变换,仍然是梳函数。\\

		为什么需借助傅里叶级数计算梳函数的傅里叶变换,而非直接计算?\\

		梳函数是周期函数,不满足\textit{绝对可积}条件,无法计算傅里叶变换。分解为傅里叶级数后,每个分量都满足\textit{绝对可积}。\\

		不仅是梳函数,所有的周期函数,都应该通过这个方式计算傅里叶变换。

		\textbf{梳函数卷积},

		\begin{align*}
			g(\tau) = \mathop{C}(x,T) * f(x) = \int \mathop{C}(\tau-x,T)f(x)dx
		\end{align*}

		梳函数翻转与自身对称,所以$\mathop{C}(-x,T) = \mathop{C}(x,T)$,及
		$$
			C(\tau-x,T) = C(x+\tau,T) = \sum_{n=-\infty}^{\infty}\delta(x + \tau - nT)
		$$

		根据$\delta$函数的选择性,
		$$
			g(\tau) = \sum_{n=-\infty}^{\infty}f(\tau - nT)
		$$

		得到一个重要结论:任意函数与梳函数卷积,相当于原函数平移,对应点求和。平移,相当于对原函数做了周期沿拓。
		\begin{figure}[H]\label{comb_replica}
			\begin{center}
				\includegraphics[width=\textwidth]{./images/comb_shift.png}
			\end{center}
			\caption{不同周期梳函数对函数的平移}
		\end{figure}

		根据上图可知,当梳函数的周期大于等于目标函数周期时,每个点对应一个函数值;否则有多个函数值,计算$g(\tau)$会出现歧义。\\

		在信号的频域中,这就是\textit{频率混叠}现象,后面会再讨论。

\subsection{逆变换证明}
	\begin{align*}
		\mathcal{F}^{-1}(\mathcal{F}(f))(x) &= \int_{-\infty}^{\infty} \hat{f}(\omega) e^{i2\pi x\omega}d\omega\\
			 &= \int_{-\infty}^{\infty} 
			 \left[
			 \int_{-\infty}^{\infty}f(t)e^{-i2\pi \omega t}dt e^{i2\pi x\omega}\right]d\omega\\
			 &= \int_{-\infty}^{\infty} f(t)
			 	\left[
			 		\int_{-\infty}^{\infty}e^{i2\pi \omega (x-t)}d\omega
			 	\right]dt
	\end{align*}

	由(\ref{delta_int_form})知,
	$
		\delta(x-t) = \int_{-\infty}^{\infty}e^{i2\pi \omega (x-t)}d\omega
	$

	$$
		\mathcal{F}^{-1}(\mathcal{F}(f))(x) = \int_{-\infty}^{\infty} f(t) \delta(x-t)dt = f(x)
	$$

	最后一步跟进$\delta$函数选择性。

\subsection{时域频域}

	时域,频域都是函数空间,通过傅里叶变换,在两个空间建立了一一映射,
	$$
		f(x) \overset{\mathcal{F}}{\rightarrow} \hat{f}(\omega) 
	$$

	我们在时域定义函数乘法运算,在频域定义卷积预算,这两个运算在$\mathcal{F}$映射下也是对应的,所以粗略来说,时域与频域如果看作群的话,两个群是\textit{同构}的。\\

	代数上,同构群认为是相同的,所以在时域上分析信号随时间变化,等价于频域上分析信号随频率的变化,而后者经常更容易一些。\\

	当然这两个空间构不成群,因为不是所有的函数都有反函数,但这并不在重要。
