%-*- coding: UTF-8 -*-
\documentclass[hpyerref,UTF8,a4paper,titlepage,12pt,oneside]{ctexbook}
\usepackage{hyperref}
\usepackage{geometry}
\usepackage{xeCJK, fontspec, xunicode, xltxtra,ulem}
\usepackage{amsthm}
\usepackage{amsmath}
\usepackage{amssymb}
\usepackage{mathrsfs}
\usepackage{mathtools}
\usepackage{commath}
\usepackage{listings}
\usepackage{float}
\usepackage{xcolor}
\usepackage{mdframed}

\graphicspath{{images/}}
\geometry{a4paper,bottom=2cm}

\title{Multi-View Stereo for Community Photo Collections}
\author{陈国庆}
\date{\today}

\bibliography{plain}

% 定理结构
\theoremstyle{definition}
\newtheorem{definition}{定义}[section]
\newtheorem{theorem}{定理}[section]
\newtheorem{corollary}{推论}[theorem]
\newtheorem{lemma}[theorem]{Lemma}
\renewcommand\qedsymbol{$\blacksquare$}

\begin{document}

% \maketitle
\tableofcontents

\href{https://hhoppe.com/mvscpc.pdf}{"Multi-View Stereo for Community Photo Collections"}是一篇2007年的文章,是稠密重建领域的经典文章,很多MVS都参考其思想。\\

MVS可以粗略分成两个阶段:视图匹配(View Matching)和区域生长(Regin Growing),本文关注的是视图匹配的工作;而区域生长则借用了他人的一个框架,这里并没有说的很细。\\

\subsection{背景}
在2007年是互联网开始迅猛发展,诞生了Flickr这种UGC图片网站,不同用户会对着同一建筑物在不同的时间、用不同的相机进行拍照,能否用这种照片重建出三维场景?\\

传统的MVS对拍照有很高的要求,比如多张照片要光照一致,相机间基线也有要求,但对UGC内容显然无法满足这些要求,其重建难度会更大。\\

仔细分析,UGC内容带来的最大的困难时图像匹配,只要匹配好,区域生长过程沿用之前算法也还ok,所以本文重点是聚焦在图像匹配上。\\

MVS需要知道相机内参,Flickr上的照片需携带EXIF,作者用\textbf{PTLens}这个商业软件抽取内参信息。\\

\subsection{View Selection}

图像匹配分为两个过程,
\begin{itemize}
	\item \textbf{Global View Selection} 从\textbf{视图}视角选择$N$个邻居
	\item \textbf{Local View Selection} 从\textbf{像素}视角选择$N$个邻居
\end{itemize}

\subsubsection*{Global View Selection}
	全局选择注意考虑到两张图片的视差要合理,我们知道视差太小会放大深度误差,视察过大会导致遮挡,所以视察\textbf{合适}的图片称为pair重建出的深度才可信。\\

	当然在特征匹配的时候用的基本都是SIFT特征,本文也不例外。但很多SIFT匹配很好的视图点也不适合重,比如
	\begin{itemize}
		\item 视图过于相似,比如都拍摄某个场景的局部,信息增益差
		\item SIFT的尺度不变可以匹配不同分辨率的图像,但分辨率差距太大的图像不适合重建
	\end{itemize}

	怎样的视察才是\textbf{合适}的呢?本文给出一个打分函数,
	$$
		g_R(V) = \sum_{f\in F_V \cap F_R}w_N(f)\cdot w_s(f)
	$$

	$R$是参考图像,$V$是邻域视图,$F_V,F_R$是视图$V,R$对应的SIFT特征集合,

	$$
		w_N(f) = \prod_{V_i,V_j\in N\atop s.t. i\ne j}w_\alpha(f,V_i,V_j)
	$$

	其中,
	$$
		w_\alpha(f,V_i,V_j) =\min((\alpha/\alpha_{max})^2,1)
	$$

	$\alpha$是两个视图相机光心与特征$f$对应3D点的夹角,$\alpha_{max} = 10^\circ$是阈值,作者认为$10^\circ$是一个理想值。\\

	要注意到平方项,如果$\alpha < 10^\circ$,平方会使得$\alpha/\alpha_{max}$迅速变小,从而乘法更小的视差;如果$\alpha/\alpha_{max}$

\bibliography{math}
\end{document}