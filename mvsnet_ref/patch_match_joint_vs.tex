\section{Joint View Selection}

\textbf{PatchMatch Based Joint View Selection and Depthmap Estimation}\footnote{\url{https://openaccess.thecvf.com/content_cvpr_2014/papers/Zheng_PatchMatch_Based_Joint_2014_CVPR_paper.pdf}} 是CVPR2014的一篇文章,主要是思路是通过变分推断来求解多视图视觉问题。\\


前面的文章主要针对\textit{双目视觉},包括两部分工作,
\begin{itemize}
	\item 根据参考图(reference image)确定几张目标图(source images),主要是参考一些规则,比如基线远近灯
	\item 图像通过极线校正后,计算每个点的视差,这里面又分几个步骤;关于局部计算的窗口,上文又提出了\textbf{slanted window},同时会用PathcMatch的方法传播这个窗口计算以降低计算量
\end{itemize}


本文与此不同的是,针对\textit{多视图视觉},对应两个部分为,
\begin{itemize}
	\item source images的选择(称为\textbf{view selection})集成进算法迭代过程,不再是一个独立的过程
	\item 通过变分推断同时求解出每个点的深度和对应的source images
\end{itemize}

整体来说将基于各种规则的视差计算问题,转化为概率分布,再通过变分推断进行求解。

\subsection{模型假设}

参考图$X^{ref}$上的点p,对应的支持窗$X_p^{ref}$,经过单应变换到目标图像$X^m$对应的支持窗$X^m_p$,那么$X^m_p$“可见度”有多大呢?\\

如果$p$点在相机$m$中被遮挡了,$X^m_p$实际是不可见的,即使$p$没有被遮挡,但相机$m$与参考相机的视角较大,导致$X^m_p$与$X_p^{ref}$像素差异已经非常大,则也认为$X^m_p$的可见度很小。\\

所以“可见度”也可认为目标窗口与参考窗口之间的某种相似度,比如传统MVS里面的RGB相似度。\\

支持窗口是一个\textbf{fronto-parallel}平面,不同深度的$X_p^{ref}$变换得到的$X^m_p$是不同的,也就是说$X^m_p$也是$p$反投影深度$d$的函数。\\

本文作者所用符号异于常人,

\begin{itemize}
	\item $l$代表$p$点坐标,意义是某行中的第$l$列,但把行给省略了,$X_l^{ref} = X_p^{ref}$
	\item $\theta_l$代表$p$点反投影深度
	\item $Z_l^m \in\{0,1\}$是一个指示变量,代表点$p$是否在$X^m$中可见
\end{itemize}

给出下面的“可见度”定义,

\begin{equation}
	P\left(X_l^m\mid Z_l^m, \theta_l,X_l^{ref} \right) = 
	\begin{cases}
		\frac{1}{NA}\exp{-\frac{\left(1-\rho^m_l\right)^2}{2\sigma^2}}\quad& \text{if } Z_l^m=1\\
		\frac{1}{N}\mathcal{U}\quad & \text{if } Z_l^m=0
	\end{cases}\label{patch_similar}
\end{equation}

$\rho^m_l$用NCC(normalized cross correlation)定义了两个窗口之间的RGB相似度,实际就是相关系数。这个文章没有代码,无法得知具体实现方式。\\

$NA$是归一化系数,$N$是某个常数,$\sigma$也是超参数,$\mathcal{U}$是一个均值为0.5的均匀分布。\\

该定义的意义是,如果$p$点在相机$m$中可见,则目标窗口的可见性由$\rho\left(m,l,\theta_l\right)$来定义;否则,若不可见,则可见性是一个常数。\\

既然$p$都不可见了,为什么$X_l^m$可见性不是$0$而是一个常数呢?实际上,作为$0$是没有问题的,作者没有想清楚这个概率的实际意义是啥。\\

$X^{ref}_l$属于已知量,$P\left(X_l^m\mid Z_l^m, \theta_l,X_l^{ref} \right)$也记为,$P\left(X_l^m\mid Z_l^m, \theta_l\right)$,$Z_l,\theta_l$称为\textbf{隐变量},它们影响着目标窗口可见度,但又无法在样本中观察到。\\

给定一张参考图像和$M$张目标图像$\mathbf{X}=\{X^1,\dots,X^m\}$,$\boldsymbol{\theta}=\{\theta_l|l=1,\dots,L\}, \mathbf{Z}=\{Z^m|m=1,\dots,M\}$,最大化后验分布$P(\mathbf{Z},\boldsymbol{\theta}|\mathbf{X})$,便可得到$\mathbf{Z},\boldsymbol{\theta}$,从而预估出参考图像的深度图。\\

引入$Z^m_l$变量是一个非常创新性想法,无需事前对目标图像进行筛选,其全部可纳入迭代过程。\\

当匹配窗口时,按照PatchMatch的方式进行,即计算完第$l-1$个点,顺序进行第$l$个点的匹配,对同一目标相机,前后两点对应窗口的可见性$Z_{l-1}^m,Z^m_l$是有一定关系的,如果不考虑更远的点对当前点的影响,这个过程构成马尔科夫链。\\

本文的思路是通过\textbf{变分推断}结合\textbf{隐马尔科夫链}来求解这个MAP问题,这里还需讨论假设是否合理,怎么定义隐变量等问题。

\subsection{期望最大化}

	对分布$p(\mathbf{x},\mathbf{z};\theta)$,$\mathbf{z}$为隐变量,$\theta$为参数,最大化后验概率(MAP)等价与最大化似然函数,对应的优化目标为

	$$
		\mathop{Loss}(\theta) = \sum_{i}\log\sum_{z_i} p(x_i|z_i;\theta)p(z_i;\theta)
	$$

	对这种带隐变量的问题,直接对$\mathop{Loss}$求导解出$\theta$几乎是不可能的事,通常通过EM算法来优化该函数的下界,来达到最终优化$\mathop{Loss}(\theta)$的目的。\\

	构造隐变量的中间分布$q_i$($\sum q_i(z_i) = 1$),则

	\begin{align}
		\mathop{Loss}(\theta) 
		&= \sum_i\log\sum_{z_i}p(x_i,z_i;\theta)\\
		&= \sum_i\log\sum_{z_i} Q_i(z_i) \frac{p(x_i,z_j;\theta)}{Q_i(z_i)}\\
		&\geq \sum_i\sum_j Q_i(z_i) \log\frac{p(x_i,z_i;\theta)}{Q_i(z_i)}\\
		&= \mathcal{L}(\theta)
	\end{align}

	倒数第二步,$\log$是凹函数,通过Jensen不等式可得。\\

	\begin{equation}
		\mathcal{L}(\theta) = \sum_i\sum_{z_i} Q_i(z_i) \log\frac{p(x_i,z_i;\theta)}{Q_i(z_i)}
	\end{equation}\label{ELBO}

	为$\mathop{Loss}(\theta)$的下界。

	\subsubsection*{E-Step}
		选择什么样的$Q_i$才能使得下界更紧?Jensen不等式成立的条件是,
		$$
			\frac{p(x_i,z_i;\theta)}{Q_i(z_i)} = c
		$$

		所以
		$$
			p(x_i,z_i;\theta) = c Q_i(z_i)
		$$

		两边对$z$求和,并结合条件$\sum_{z_i} Q_i(z_i) = 1$,可知

		$$
			c = \sum_{z_i} p(x_i,z_i;\theta) = p(x_i;\theta)
		$$

		所以,
		\begin{align}
			Q_i(z_i) = \frac{p(x_i,z_i;\theta)}{p(x_i;\theta)} = p(z_i|x_i;\theta)
		\end{align}\label{e_step}

		所以,使得下界最紧的$Q_i$是$x_i$对中$z_i$的后验分布。
	\subsubsection*{M-Step}
		利用E-Step得到的$Q_i$,最大化$\mathcal{L}(\theta)$来得到参数$\theta$,第$t+1$次迭代,

		\begin{align}
			\theta^{(t+1)} &= \arg\max_\theta \mathcal{L}(\theta)\\
				&= \arg\max_\theta \sum_i \sum_{z_i} p(z_i|x_i;\theta^{(t)})\log p(x_i,z_i;\theta)\\
				&= \arg\max_\theta \sum_i \sum_{z_i} p(z_i|x_i;\theta^{(t)})\log p(x_i|z_i;\theta)p(z_i;\theta)\label{theta_eq}\\
		\end{align}

		第二步是将E-Step得到的$Q_i$(\ref{e_step})代入(\ref{ELBO})。$p(z_i|x_i;\theta^{(t)})$可表示为,

		$$
			p(z_i|x_i) = \frac{p(x_i|z_i)p(z_i)}{\sum_i p(x_i|z_i)p(z_i)}
		$$

		在解决具体问题是,只要明确$p(z_i;\theta),p(x_i|z_i;\theta)$的形式便可迭代求解。

\subsection{EM求解MVS}
		在本文中,$\mathbf{Z}= (Z^m|m=1,\dots,M)$为隐变量,$\theta = (\theta_1,\dots,\theta_L)$为参数;\\

		$Z^m = (Z^m_1,\dots,Z^m_L)$是一个$L$维的随机向量,$L$为图像像素数量。\\

	\subsubsection*{求解隐变量$Z^m$}

		根据(\ref{e_step}),
		$$
			Q_i(Z^m) = p(Z^m|X^m;\theta) \propto p(X^m|Z^m;\theta)p(Z^m;\theta)
		$$

		根据假设$Z^m_l$的状态转移构成了马尔科夫链,

		\begin{equation}
			p(Z^m) = p(Z^m_1)\prod_{l=2}^L p(Z^m_l|Z^m_{l-1})\label{markov_chain}
		\end{equation}

		状态转移矩阵为,
		$$
			p(Z^m_l|Z^m_{l-1}) = \begin{bmatrix}
				\gamma\quad & 1-\gamma\\
				1-\gamma\quad & \gamma
			\end{bmatrix}
		$$

		$\gamma=0.999$为超参数。

		\begin{equation}
			p(X^m|Z^m;\theta) = \prod_{l=1}^L p(X^m_l|Z^m_l;\theta_l)\label{first_expand}
		\end{equation}

		结合(\ref{markov_chain})、(\ref{first_expand})可得$q^*(Z^m)$为,

		\begin{equation}
			\boxed{
			q^{opt}_m(Z^m) \propto  p(Z^m_1)\prod_{l=2}^L p(Z^m_l|Z^m_{l-1})  \prod_{l=1}^L p(X^m_l|Z^m_l;\theta_l)
			}\label{q_z_eq}
		\end{equation}

		这与原文中的结论是一致的。

	\subsubsection*{求解参数$\theta$}

		根据文中假设,

		\begin{itemize}
			\item $p(Z^1)$为均匀分布,与参数$\theta$无关;
			\item 状态转移矩阵$p(Z^m_l|Z^m_{l-1})$与参数$\theta$无关;
			\item 由(\ref{markov_chain}),$p(Z^m)$与参数$\theta$无关;
		\end{itemize}

		原文是最小化KL距离,等价与最大化$\mathcal{L}(\theta)$,与M-Step的结果(\ref{theta_eq})一致,

		\begin{align*}
			\theta^{(t+1)} &= \arg\max_\theta \sum_i \sum_{z_i} p(z_i|x_i;\theta^{(t)})\log p(x_i|z_i;\theta)p(z_i;\theta)\\
			&= \arg\max_\theta \sum_i \sum_{z_i} Q_i(z_i;\theta^{(t)})\log p(x_i|z_i;\theta)p(z_i;\theta)\\
			&= \arg\max_\theta \sum_{m=1}^M \sum_{Z^m} q^*_m(Z^m;\theta^{(t)})\log p(X^m|Z^m;\theta)p(Z^m)\\
			&= \arg\max_\theta \sum_{m=1}^M \sum_{Z^m} q^*_m(Z^m;\theta^{(t)})\log p(X^m|Z^m;\theta)\\
			&= \arg\max_\theta \sum_{m=1}^M \sum_{Z^m_l} 
				\left[ 
					\prod_{l=1}^L q^*_m(Z^m_l;\theta^{(t)}_l)\sum_{l=1}^L\log p(X^m_l|Z^m_l;\theta_l)
				\right]
		\end{align*}	

		考虑第$l$个参数,

		\begin{align}
			\theta^{(t+1)}_l 
			&=\arg\max_\theta \sum_{m=1}^M \sum_{Z^m} 
			\left[ 
				\prod_{l=1}^L q^*_m(Z^m_l;\theta^{(t)}_l) \right]\cdot
				\log p(X^m_l|Z^m_l;\theta_l)
			\\
			&=\arg\max_\theta \sum_{m=1}^M 
				\left[
					\prod_{l=1}^L q^*_m(Z^m_l=1;\theta^{(t)}_l)
				\right]\log p(X^m_l|Z^m_l=1;\theta_l)\label{em_theta}
		\end{align}

		对比一下原文中的结论,

		\begin{equation}
			\theta^{opt}_l = \arg\max_{\theta^*_l}\sum_{m=1}^Mq(Z^m_l = 1)\ln P(X^m_l|Z^m_l=1,\theta_l=\theta^*_l)
		\end{equation}
		
		二者差异是第一项图片$m$的权重,我们是$\prod_{l=1}^L q^*_m(Z^m_l=1;\theta^{(t)}_l)$,原文是$q(Z^m_l = 1)$。我们推导并无问题,原文结果是错误的。\\

		猜测作者的结果参考\textit{混合高斯}模型,在\textit{混合高斯}中,隐变量$z_i$是单变量,对应上述结果;而在MVS场景,$z_i$是多变量,对应我们得结果。\\

		$p(X^m_l|Z^m_l=1;\theta_l)$在(\ref{patch_similar})被定义,代入可得

		\begin{equation}
		\boxed{
			\theta^{(t+1)}_l = \arg\min_\theta \sum_{m=1}^M\left[
					\prod_{l=1}^L q^*_m(Z^m_l=1;\theta^{(t)}_l)
				\right]\left(1- \rho_l^m\right)^2
				}\label{correct_theta}
		\end{equation}

		也不同于原文结果,
		\begin{equation}
			\theta^{opt}_l = \arg\min_\theta \sum_{m=1}^M q(Z^m_l=1)\left(1- \rho_l^m\right)^2 \label{paper_theta}
		\end{equation}

		后面基于采样计算$\theta^{opt}_l$自然也就不再正确。

	\subsubsection*{HMM假设}

		如前讨论,对图片$m$,本文假设$p(Z^m_l|Z^m_{l-1})$构成隐马尔科夫链,看一下对应的要素,

		\begin{itemize}
			\item 隐变量为$Z^m_l$,对应的状态集合$Q=\{0,1\}$
			\item 转移矩阵A,前面已定义
			\item 观测状态为$p(X^m_l|Z^m_l;\theta)$,观测集合$O = \{p(X^m_l|Z^m_l;\theta)\}$($l=1,\dots,L$),观测状态是一个概率,并非是一个离散值
			\item 观测矩阵B,观测状态不可数,无法定义
			\item 初始向量$\pi$,可随机初始化
		\end{itemize}

		因为无法定义观测矩阵,所以这并不是一个well-defined的隐马模型。\\

		well-defined的HMM模型,对确定的观测序列,可通过前向后向算法来求解某时刻下的状态概率,但这依赖观测矩阵B,若B无法定义,则算法无法执行。\\

		行文至此,作者突然提出要通过前后向算法计算$q(Z^m)$,并给出如下式子,

		\begin{align*}
			q(Z_l) &= \frac{1}{A}\alpha(Z_{l})\beta(Z_l)\\
			\alpha(Z_{l}) &= p(X_l|Z_l,\theta_l)\sum_{Z_{l-1}}\alpha(Z_{l-1})P(Z_l|Z_{l-1})\\
			\beta(Z_l) &= \sum_{Z_{l+1}}\beta(Z_{l+1})P(X_{l+1}|Z_{l+1},\theta_{l+1})P(Z_{l+1}|Z_l)
		\end{align*}

		作者可能犯了几个错,
		\begin{enumerate}
			\item $p(Z_l)$是马尔科夫链,$q(Z_l)$并不是,作者可能迷糊了
			\item $q(Z_l)$已经在(\ref{q_z_eq})求解出来了
			\item 作者误把$P(X_{l+1}|Z_{l+1},\theta_{l+1})$当作B,进行了前后向计算,这实际是观测状态
		\end{enumerate}

	\subsubsection*{HMM是否需要}

		非well-defined的HMM,能否通过EM求解?\\

		通过EM算法求解混合高斯时,需要\textbf{Mean-Field}假设,即隐变量相互独立;但本文中,做的是\textbf{马尔科夫依赖}假设,即隐变量只依赖前一状态,并非完全独立。\\

		无论作何种假设,目的都是为了求解隐变量的联合分布,通过马尔科夫依赖也足以求解,是不是马尔科夫链并不重要。\\

		更无需关注某个状态在观测序列中的概率,也不需去执行前后向算法。\\

		确定$q^{opt}_m(Z^m), q^{opt}_l(\theta_l)$的具体形式,便可以执行E-M算法。
	
\subsection{合理性\&实验效果}

	从实验部分看,缺少跟大多数算法的全面比对,在小数据集上效果也不明显,也没有给出实现代码,上文提到的各种问题也无法知道具体编码中是怎样处理。\\

	理论部分缺少细节且存一些严重的问题,在HMM部分甚至有乱写倾向。\\

	我们无法了解内部代码的实现方式,但做不出效果感觉也在情理之中。\\

	此文开始提到他们用变分推断求解,我花费大量时间通过变分推断去推导结果,最终发现他们实际用的是EM算法。

