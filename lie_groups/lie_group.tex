\section{李群李代数}

群是集合,流形也是集合,把二者结合为一则为\textbf{李群},所以李群是群,也是微分流形。\\

当然除了简单概念结合,还需要群乘法是光滑映射。\\

两个李群$G,G^\prime$,如果存在光滑的同态映射,则称为\textbf{李群同态}。\\

如果$G,G^\prime$同构且微分同胚,则称为\textbf{李群同构},李群同构确保两个李群的代数结构和几何结构都“完全一样”。\\

前面提到的单参微分同胚群为李群,机器人运动姿态,自动驾驶中各传感器的姿态基3D重建中的相机姿态,也都是李群。

\subsubsection*{左平移}
利用群乘法定义李群$G$中的左平移,
$$
	L_g:= gh
$$

则容易知道,左平移有几个特点,
\begin{itemize}
	\item $L_e$是恒等元
	\item $L_{gh} = L_g\circ L_h$
	\item $L_g^{-1} = L_{g^{-1}}$
	\item $L_g$是$G\mapsto G$的微分同胚
\end{itemize}

\subsubsection*{左不变矢量场}

	如果$G$上某个矢量场$\bar{A}$在左平移下不变,
	$$
		L_{g*} \bar{A} = \bar{A}, \forall g \in G
	$$
	则称为左不变矢量场,如此可得到左不变矢量场等价表示,
	$$
		 L_{g*}\bar{A}_h = \bar{A}_{gh}
	$$

	不难验证,$G$上所有左不变矢量场集合(记为$\mathcal{L}$)是一个矢量空间,并且有下面定理,
	\theorem{$G$在单位元的切空间$V_e$与$\mathcal{L}$线性同构。}




\subsubsection*{李代数}

李群也有切空间,李群存在特殊元素,单位元$e$,所以$T_eM$是一个特殊的切空间。\\

还需要给$T_eM$武装上一个二元运算$[,]$,称为\textbf{李括号} ,需要满足下面几个条件,

\subsubsection*{单参子群}
	光滑映射$\gamma$,如果满足下面条件($G$为李群),
	$$
		\gamma: \mathbb{R}\mapsto G, \gamma(r+t) = \gamma(s)\gamma(t)
	$$

	则称为李群$G$上的\textbf{单参子群};$\gamma$实际上是$\mathbb{R}$到$G$的同态映射,也是$G$上的曲线。\\

	根据单参子群定义容易验证,$\gamma(0) = e$,且的确满足群定义的4个条件。

\begin{itemize}
\item 
\end{itemize}