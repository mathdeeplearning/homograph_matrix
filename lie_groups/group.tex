\section{群}

$G$是一个集合,在$G$上定义\textit{群乘法}($\cdot$),$G$及对应的乘法($\cdot$)满足下面4个条件,则称$(G,\cdot)$为群,

\begin{itemize}
	\item \textbf{封闭性},$g\cdot h \in G,\quad \forall g,h \in G$
	\item \textbf{结合律},$g\cdot(h\cdot k) = (g\cdot h)\cdot k$
	\item \textbf{单位元},$\exists e\in G$,$g\cdot e = e\cdot g = g,\quad \forall g \in G$
	\item \textbf{逆元},$\exists g^{-1} \in G$,使得$g\cdot g^{-1} = e, g^{-1}\cdot g = e,\quad \forall g \in G$
\end{itemize}

群定义不要求满足\textit{交换律},群乘法($\cdot$)通常可以省略,$g\cdot h$记为$gh$,$(G,\cdot)$记为 $G$。\\

把$\mathbb{R}$作为一个集合,群乘法定义为实数加法,则$(\mathbb{R},+)$构成一个群;若把群乘法定义为实数乘法,则无法构成群,因为0没有逆元。\\

这里一定要清楚,无论群元是什么类型,群上只定义了一种运算,就是群乘法,对群$(\mathbb{R},+)$而言,除了加法其他实数运算都跟群没有关系。

\subsubsection*{\textbf{一般线性群}$GL(n)$}
$$
	GL(n) = \left\lbrace A \in M_{n\times n}|\det(A) \ne 0\right\rbrace
$$
是定义在$\mathbb{R}$上的,$n$阶可逆矩阵的集合,群乘法定义为矩阵乘法。

\subsubsection*{\textbf{特殊正交群}$SO(n)$}

$$
	SO(n) = \left\lbrace R \in M_{n\times n}|RR^T = 1,\det(R) =1\right\rbrace
$$
是单位旋转矩阵的集合,$SO(2),SO(3)$对应二维及三维旋转矩阵。

\subsubsection*{\textbf{特殊欧式群}$SE(3)$}
	通常通过齐次坐标,将旋转、平移用一个$4\times 4$矩阵来表示,
	$$
		T = \begin{bmatrix}
		R_{3\times 3} &\quad \mathbf{v}_{3\times 1}\\
		\mathbf{0}_{1\times 3} &\quad 1
		\end{bmatrix}
	$$
	其中$R \in SO(3)$是一个旋转矩阵,$\mathbf{v}$是一个平移向量,很明显$T$是可逆的,所以$T \in GL(4)$,因为其形式特殊又称为$SE(3)$或,特殊欧式群。\\

	群乘法依然定义为矩阵乘法,可以验证该定义封闭。

\subsubsection*{群同态}
	群同态是两个群之间保群乘法的映射,具体来说,映射$\pi$
	$$
		\phi: G \mapsto G^\prime
	$$
	满足
	$$
		\phi(gh) = \phi(g)\phi(h)
	$$

	比如,$G$为$n\times n$矩阵集合,群乘法定义为矩阵相乘;$G^\prime = \mathbb{R}$,$\phi$为取行列式运算,根据线性代数可知,
	$$
		\det(AB) = \det(A)\det(B)
	$$

	所以$\det$是$G$与$\mathbb{R}$之间的同态映射。\\

	同态映射构造了两个群乘法之间的对应关系,可以借助$G$中的操作构造$G^\prime$中的操作。\\

	比如说,向量空间$V$与某群$G^\prime$之间存在同态映射$\phi$,如何在$G^\prime$上定义微小增量呢?\\

	我们知道在向量空间$V$上定义增量为$x + dx$,可借助同态映射$\phi$,
	$$
		\phi(x+dx) = \phi(x)\phi(dx)
	$$

	来完成$G^\prime$上增量定义。\\

	实际上,李群导数正是借助李代数与李群之间的同态映射构造增量求解导数,这个映射为指数映射。

\subsubsection*{群同构}
	如果同态映射是一一的且是满的,相当于存在逆映射,则称两个群同构。\\

	比如$SO(2)$,为2D平面旋转矩阵群,群员可表示为,
	$$
		A(\theta) = \begin{bmatrix}
			\cos\theta &\quad -\sin\theta\\
			\sin\theta &\quad \cos\theta
		\end{bmatrix}
	$$

	定义映射,
	$$
		f: SO(2) \mapsto \mathbb{R}
	$$

	$$
		f(A(\theta)) = \theta
	$$

	根据旋转矩阵的性质容易验证,$f$是同态映射,并且旋转矩阵与$\theta$是一一对应的,所以$SO(2)$与$\mathbb{R}$同构。\\

	从结构上来讲,两个群同构则认为二者“完全相同”,比如旋转角$\theta$与旋转矩阵是一回事。\\

	对固定的$g\in G$,定义映射,
	$$
		\phi(h) = ghg^{-1}
	$$

	容易验证这个映射是$G\mapsto G$的同构,称为\textbf{伴随同构}。